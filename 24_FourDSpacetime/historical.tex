\begin{historical}
The idea that the dimensionality of physical space might be constrained by necessities predates the formal development of modern physics. Gottfried Wilhelm Leibniz suggested in the \emph{Discourse on Metaphysics} (1686) that the actual world should be understood as the one “simplest in hypotheses and richest in phenomena,” implicitly framing dimensionality as subject to selection principles. In the 18th century, Immanuel Kant proposed that Newton’s inverse-square law implied the three-dimensionality of space, although later assessments — notably by John D. Barrow — reversed this causal inference: the inverse-square law follows from spatial geometry, not the reverse.

A more analytic approach began with Paul Ehrenfest in 1920, who showed that classical orbit stability requires exactly three spatial dimensions when time remains one-dimensional. In higher spatial dimensions ($N > 3$), the effective gravitational potential falls off too quickly to maintain closed, bounded orbits. Ehrenfest also noted that wave propagation becomes degenerate in even-dimensional spaces and distorted in $N = 5 + 2k$, where $k \in \mathbb{N}$. This links the preservation of coherent wavefronts to dimensional parity. In 1922, Hermann Weyl observed that the action formulation of Maxwell’s theory acquires natural geometric expression only in four dimensions, further reinforcing the specificity of $D = 4$.

Stability arguments were extended in quantum mechanics. In 1963, Frank R. Tangherlini generalized the Schrödinger equation and showed that hydrogen-like atoms lose their bound states in $N > 3$. The potential wells become too shallow for discrete energy levels, causing electrons to collapse into nuclei or escape entirely. More recently, Max Tegmark examined the implications of altering the number of temporal dimensions. When $T \ne 1$, the initial value problem for hyperbolic differential equations becomes ill-posed, destroying predictive dynamics and making the evolution of physical systems non-deterministic. For $T > 1$, proton decay becomes generically kinematically allowed unless tightly suppressed.

The anthropic character of these arguments was recognized early, despite predating the formal vocabulary of anthropic reasoning. They point to a dimension count — $N = 3$, $T = 1$ — that appears simultaneously necessary for gravitational, electromagnetic, quantum, and biological structure. In this view, four-dimensional spacetime is not merely observed but required for complexity to exist.

Further thermodynamic and cosmological constraints have also been proposed. Wei-Xiang Feng argued that self-gravitating gas spheres in equilibrium are only stable in $D = 4$ when the cosmological constant is positive but small. For masses above a critical bound, such configurations collapse into black holes or evaporate dynamically. These arguments suggest that even global equilibrium conditions depend delicately on the dimension of spacetime.

Although isolated counterarguments exist — such as James Scargill’s 2019 claim that scalar-field networks might allow for complex structure in $N = 2$ — no known models in $N \ne 3$ and $T \ne 1$ have been shown to support the full range of empirically verified interactions, nor the conditions required for stable observers.
\end{historical}
