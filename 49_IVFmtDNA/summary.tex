Mitochondrial DNA (mtDNA) exists separately from nuclear DNA, containing 37 genes essential for cellular energy production through oxidative phosphorylation. This genetic material is inherited almost exclusively from the oocyte during fertilization, as sperm contribute negligible cytoplasmic content. mtDNA's vulnerability to mutations — 10-100 times higher than nuclear DNA — stems from its lack of protective histones, limited repair mechanisms, and proximity to reactive oxygen species. Standard in vitro fertilization preserves this maternal inheritance pattern, potentially transmitting pathogenic mtDNA mutations to offspring. Mitochondrial replacement therapies address this by transferring nuclear material from an affected oocyte to donor cytoplasm containing healthy mitochondria, using either maternal spindle transfer before fertilization or pronuclear transfer after fertilization but before nuclear fusion. While rare reports suggest paternal mtDNA transmission, these typically represent nuclear mitochondrial DNA segments (NUMTs) misidentified during sequencing rather than actual biparental inheritance.