\begin{historical}
The muon was discovered in 1936 by Carl D. Anderson and Seth Neddermeyer during cosmic-ray experiments, an unexpected find because it seemed to serve no clear role in nature. Originally called the “mesotron,” it was thought to mediate forces inside the nucleus, but that idea soon fell apart when it failed to behave like the then-known pions.

In the subsequent years, research revealed muons are produced when cosmic rays strike the upper atmosphere, creating showers of secondary particles that include pions. These pions decay into muons, which in turn decay into electrons and neutrinos. Early measurements in the 1930s and 1940s hinted at more muons reaching sea level than non-relativistic predictions could explain. By the late 1940s, experiments by Bruno Rossi and others connected these findings to relativistic time dilation, solidifying the muon's role as an early test case for Einstein’s special relativity.
\end{historical}
