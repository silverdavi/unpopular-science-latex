Muons created by cosmic rays colliding with the upper atmosphere provide direct evidence for time dilation. With a rest-frame lifetime of approximately 2.2 microseconds and traveling close to light speed, classical physics predicts these particles should decay before reaching Earth's surface. Instead, detectors routinely observe muons at sea level. Special relativity explains this observation: from Earth's reference frame, the muons' time runs slower by a factor of γ (approximately 10-50 depending on energy), extending their lifetime enough to reach ground level. From the muon's perspective, relativistic length contraction reduces the distance traveled.
