\begin{historical}
At the turn of the 20th century, Karl Pearson encountered inconsistencies when analyzing grouped data, particularly in contingency tables where trends in combined data did not match those seen within subgroups. While he acknowledged the issue, he did not isolate its general mechanism. The phenomenon remained an implicit anomaly until 1951, when Edward H. Simpson published a short but influential note in which he formalized the algebraic conditions under which aggregated data can misrepresent subgroup trends. Simpson demonstrated that marginal associations could reverse direction when a third variable — now understood as a confounder — was not conditioned upon.

Although Simpson's paper initially received little attention, the result gained visibility in the 1970s as debates over confounding and causal inference intensified. Its relevance was cemented by a widely discussed case from the 1973 University of California, Berkeley graduate admissions cycle. The data showed that women had lower overall acceptance rates than men, but department-level breakdowns revealed no bias — and in many cases, a slight preference toward female applicants. The reversal was due to differential application rates to departments with varying selectivity. The case made Simpson’s analysis a touchstone for understanding aggregation bias.

Since then, the paradox has become central to methodological debates in epidemiology, social science, and machine learning. It illustrates the necessity of conditioning on relevant variables, and the dangers of interpreting marginal statistics without considering how data are partitioned. Simpson's result now serves not only as a technical lemma in probability theory, but as a conceptual warning about the limits of inference from observational data.
\end{historical}
