\begin{technical}
{\Large\textbf{Orbital Symmetry and the Quantum Constraint on Reactivity}}\\[0.7em]

\noindent\textbf{Symmetry in the Schrödinger Framework}\\[0.5em]
The Schrödinger equation \( \hat{H}\psi = E\psi \) governs the electronic structure of molecules. When the molecular Hamiltonian \( \hat{H} \) commutes with a symmetry operator \( \hat{S} \), the system’s eigenfunctions must reflect that symmetry:
\[
[\hat{H}, \hat{S}] = 0 \quad \Rightarrow \quad \hat{S}\psi = \lambda\psi.
\]
This imposes a conserved quantum label (irreducible representation) on the wavefunction throughout any geometry-preserving deformation. In a concerted reaction such as a pericyclic transformation, where all bond changes occur in a cyclic, symmetry-retaining transition state, this leads to a constraint: only reactions that preserve orbital symmetry continuity are allowed. This is the foundation of the Woodward–Hoffmann rules.

\noindent\textbf{Phase Symmetry and Frontier Orbitals}\\[0.5em]
The molecular orbitals (MOs) of conjugated systems can be described as linear combinations of atomic p orbitals. For a linear polyene with \( n \) p orbitals, the \( k^\text{th} \) MO has the form:
\[
\Psi_k = \sum_{j=1}^n \sin\left( \frac{\pi k j}{n+1} \right) p_j,
\]
where \( p_j \) are orthogonal atomic orbitals. The phase of the terminal lobes in the HOMO (highest occupied MO) determines the allowed mode of bond formation. For example:
\begin{itemize}
\item Butadiene (4 π electrons): HOMO has opposite terminal phases → conrotatory closure aligns lobes → allowed thermally.
\item Hexatriene (6 π electrons): HOMO has same terminal phases → disrotatory closure preserves overlap → allowed thermally.
\end{itemize}
These rules emerge not from empirical fits but from the symmetry character of the MOs under conserved operations (like a \( C_2 \) axis or mirror plane in the transition state).

\noindent\textbf{General Selection Rule}\\[0.5em]
Let \( N \) be the number of electron pairs involved, and \( A \) the number of antarafacial components (where bonding occurs from opposite faces of a π system). The thermal symmetry selection rule is:
\[
\text{Reaction allowed if } N + A \equiv 1 \pmod{2}.
\]
For photochemical reactions (excited-state initiated), the parity condition inverts:
\[
N + A \equiv 0 \pmod{2}.
\]
This criterion applies to electrocyclic reactions, sigmatropic shifts, and cycloadditions alike. The electron count includes both π electrons and migrating σ pairs, depending on context.

\noindent\textbf{Correlation Diagrams and Symmetry Conservation}\\[0.5em]
A more formal approach uses correlation diagrams, where each MO is labeled by its symmetry character under a conserved symmetry operation (e.g., S for symmetric, A for antisymmetric). The MOs of reactants and products are then connected across the reaction coordinate:
\begin{align*}
\text{Butadiene:} & \quad \Psi_1\ (A),\ \Psi_2\ (S) \\
\text{Cyclobutene:} & \quad \pi\ (A),\ \sigma\ (S)
\end{align*}
Under a \( C_2 \) axis (conrotatory path), the symmetry labels match, and the transformation preserves orbital occupation → allowed. Under a mirror plane (disrotatory path), the correlation fails (occupied orbital would map to unoccupied antibonding orbital) → forbidden.

\noindent\textbf{Sigmatropic Shifts and Topological Classifications}\\[0.5em]
Sigmatropic shifts involve the migration of a σ-bonded atom across a delocalized π system. The cyclic transition state contains the migrating group plus the π system—usually a 6-electron or 4-electron arrangement. For [i,j] shifts, the shift is thermally allowed if:
\[
i + j \equiv 2 \pmod{4}.
\]
A [1,5]-hydrogen shift is allowed (6 electrons), while a [1,3] shift is forbidden suprafacially (4 electrons, antiaromatic topology) and only feasible antarafacially, which is usually sterically blocked.


\noindent\textbf{References:}\\
Woodward, R. B., Hoffmann, R. (1965). \textit{JACS}, 87, 395–397.\\
\end{technical}
