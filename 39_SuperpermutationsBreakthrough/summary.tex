A significant breakthrough in the superpermutation problem originated from an unlikely source: an anonymous 4chan post responding to a question about anime episode viewing orders. Superpermutations are strings containing every possible ordering of n symbols as substrings. For years, mathematicians believed the minimal length followed the pattern of factorial sums observed in small cases. The anonymous poster derived a rigorous lower bound: L(n) ≥ n! + (n-1)! + (n-2)! + n - 3, modeling the problem as path optimization through a permutation graph. This proof remained obscure until 2018 when mathematician Robin Houston rediscovered it, leading to the disproof of the long-standing conjecture and establishing new bounds on this combinatorial problem — with the original derivation still officially credited to "Anonymous 4chan Poster."
