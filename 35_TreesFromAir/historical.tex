\begin{historical}
In the early seventeenth century, Jan Baptista van Helmont conducted an experiment that would later become emblematic of early quantitative biology. He planted a willow sapling in a weighed quantity of dry soil, supplied it only with water, and allowed it to grow for five years. At the end of the experiment, he found that the tree had gained over 70 kilograms in mass, while the soil had decreased by less than 60 grams. From this, he concluded — correctly in direction though not in mechanism — that the tree’s substance did not come from the soil.

Van Helmont identified water as the key source of mass, unaware of the role of atmospheric gases. His result was significant for shifting scientific attention away from Aristotelian elemental explanations and toward empirical measurement. The idea that a tree might be built from intangible substances posed a conceptual challenge to early chemistry, which had yet to recognize air as chemically active.

In the late eighteenth century, Joseph Priestley and Jan Ingenhousz discovered that plants could "restore" air that had been "damaged" by combustion or respiration. Ingenhousz, in particular, demonstrated that this process required light and occurred only in green parts of plants. These observations hinted at a connection between sunlight, plant matter, and atmospheric gases.

By the mid-nineteenth century, Julius von Sachs and others had established that plants produce starch in the presence of light and that carbon dioxide is the source of carbon in organic compounds. Quantitative combustion analysis allowed chemists to determine the proportions of carbon, hydrogen, and oxygen in plant tissues, confirming that nearly all structural biomass derived from these three elements.

In the twentieth century, isotopic labeling techniques enabled direct tracing of carbon atoms from \(\mathrm{CO}_2\) into plant tissues, definitively establishing air as the origin of most biomass. Experiments using \(^{14}\mathrm{C}\)-labeled carbon dioxide showed its incorporation into sugars, cellulose, and lignin. By this time, the full chemical sequence of photosynthesis had been elucidated, and the atmospheric origin of plant mass was no longer speculative but mechanistically understood.

The idea that trees are constructed from air — while counterintuitive — thus emerged gradually through centuries of refinement. It required shifts in measurement capability, chemical theory, and conceptual frameworks regarding what "substance" means in biological growth. What began as a puzzle in natural philosophy became a quantitative fact of plant physiology.
\end{historical}
