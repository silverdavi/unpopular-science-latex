A tree’s material body — the wood, leaves, and branches it accumulates year by year — is not extracted from the ground in the way stones or metals are quarried. Its dry mass arises from elements that were once distributed in dilute form throughout the atmosphere and hydrosphere. The key components of this mass — carbon, oxygen, and hydrogen — enter not through mineral substrates but through invisible flows: air, water, and sunlight. A tree is not built from what it stands on, but from what passes through it.

Although visually and mechanically tied to the soil, the structure of a tree records processes that unfold mostly above ground. The mass that persists after all water is removed — the dry matter — is composed primarily of carbon atoms originally fixed from atmospheric \(\mathrm{CO}_2\). These atoms were drawn down through the stomata of leaves, diffused through mesophyll tissue, and incorporated into sugar molecules via light-powered biochemical cycles. The resulting solid form is the memory of many such molecular events.

The notion that trees “grow out of the earth” conflates anchorage with origin. The soil does provide essential ions and mechanical stability, but its contribution to the actual mass is minor. Most of what endures in a dried trunk — cellulose, lignin, hemicellulose — was not once part of the earth beneath it, but of the air surrounding it. Even the verticality of a tree, its rise toward the sky, is materially made possible by the intake of that sky’s gaseous contents.

The central process enabling this conversion is photosynthesis. It is not a singular reaction, but a layered sequence of energy transduction and molecular reconfiguration. The first phase occurs in the chloroplasts of leaf cells, where chlorophyll pigments absorb incoming photons. These photons elevate electrons to higher energy states, dislodging them from their atomic orbitals and initiating a cascade of electron transfers through the thylakoid membrane.

This chain of transfers generates two critical energy carriers: ATP (adenosine triphosphate) and NADPH (nicotinamide adenine dinucleotide phosphate). These molecules store the electromagnetic energy harvested from light and shuttle it into the chemical domain. In the aqueous interior of the chloroplast — the stroma — this stored energy is used to convert inorganic carbon into organic intermediates.

The fixation of carbon takes place in the Calvin–Benson cycle. Atmospheric \(\mathrm{CO}_2\) diffuses into leaf tissue and reacts with ribulose bisphosphate, a five-carbon sugar, under the catalytic action of the enzyme Rubisco. The resulting six-carbon intermediate is immediately split into three-carbon molecules — triose phosphates — that serve as fundamental building blocks for carbohydrates. These triose units are reassembled into glucose and other hexoses, which in turn feed biosynthetic pathways across the plant.

Once synthesized, these sugars are exported from the site of fixation. Through the phloem — a network of conductive tissues — they are distributed to growing regions: root tips, shoot apices, developing leaves, and the vascular cambium. At the cambium, a cylindrical layer of dividing cells just beneath the bark, the imported carbohydrates are used to construct structural macromolecules.

Cellulose, hemicellulose, and lignin form the principal constituents of wood. Cellulose assembles into long, unbranched chains that crystallize into fibrils, giving tensile strength to cell walls. Hemicellulose binds these fibrils into a cohesive matrix, while lignin — a complex phenolic polymer — fills the spaces between them, adding compressive strength and water resistance. These polymers do not exist as free-floating products; they are laid down in precise geometric arrangements within the expanding walls of growing cells.

Tree growth is not an inflationary process. It is a spatially organized addition of new cells, localized in specific generative zones. At the vascular cambium, cell division proceeds laterally, producing xylem cells toward the center and phloem cells outward. This radial expansion creates the familiar pattern of growth rings. Each ring corresponds to a cycle of photosynthetic capture and biosynthetic deposition.

Elongation occurs at the apical meristems, where undifferentiated cells divide and specialize into tissue types. These regions at the tips of roots and shoots coordinate patterning, orientation, and organogenesis. As cells expand and walls thicken, the imported sugars are converted into permanent structure. The tree grows by building new matter, atom by atom, layer by layer — guided by developmental constraints, resource availability, and physical principles of load-bearing and fluid transport.

Hydrogen atoms in the biomass originate almost entirely from water. Water is absorbed by roots and pulled upward through the xylem under tension. Though over 99\% of it eventually evaporates through stomatal pores, a small fraction is chemically incorporated into organic molecules. This hydrogen forms part of the fixed structure, bound into carbohydrates and lipids.

Water’s functional role extends beyond hydrogen donation. It serves as a solvent for ions, a medium for transport, and a buffer against temperature fluctuations. It enables the tree’s biochemical metabolism without being a primary contributor to its dry weight. What remains after desiccation is not water but the elements it helped mobilize and bind.

Oxygen atoms in biomass come from both \(\mathrm{CO}_2\) and \(\mathrm{H}_2\mathrm{O}\). During photosynthesis, water molecules are split to provide electrons, releasing molecular \(\mathrm{O}_2\) as a byproduct. Some oxygen atoms are retained in the structure of sugars and polymers, forming hydroxyl, carboxyl, and ether linkages. The high oxygen content of wood — about 40 to 45 percent by weight — reflects this dual origin.

Mineral ions absorbed from the soil are essential but contribute little to total mass. Nitrogen, phosphorus, potassium, calcium, magnesium, and micronutrients serve catalytic and regulatory roles. They enable enzymatic function, membrane potential maintenance, and nucleic acid stability. However, their aggregate proportion in dry matter is often less than 5 percent. They are facilitators, not substrates.

When all water is removed from a tree, what remains is a carbon-rich composite of organic polymers. Cellulose (C\(_6\)H\(_{10}\)O\(_5\))\(_n\), lignin, and related molecules form a lattice of energy-stored mass, chemically stabilized and mechanically resilient. This material reflects a long history of atomic coordination — gases captured, energy converted, atoms arranged.

The resulting structure is not a byproduct of soil consumption. It is a physical record of electromagnetic energy transformed into covalent bonds. Each gram of wood contains photons absorbed years prior, transmuted into molecular architecture. The tree’s height, girth, and density are not residues of what it has drawn from the earth, but signatures of what it has assembled from air and light.

Richard Feynman remarked that trees are “made of air,” not as a metaphor but as a precise physical statement. When a tree burns, the carbon returns to the atmosphere, and the stored sunlight is released as heat. What remains as ash is the minor residue of earthbound elements. This framing shifts the perspective: trees are not extracted from ground but condensed from flow. They are equilibrium-defying artifacts of solar patterning and atmospheric participation.

