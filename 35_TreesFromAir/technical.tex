\begin{technical}
{\Large\textbf{Carbon Fixation and Mass Accumulation in Trees}}\\[0.7em]

\noindent\textbf{Introduction}\\[0.5em]
Trees accumulate mass primarily through the fixation of atmospheric carbon dioxide using energy derived from sunlight. This section provides the formal chemical and energetic description of how light-driven electron transfers power the synthesis of biomass from gaseous carbon inputs, resulting in stable, dense organic material over time.

\noindent\textbf{Light-Driven Reactions}\\[0.5em]
In the light-dependent reactions of photosynthesis, photons are absorbed by chlorophyll molecules in photosystems I and II, embedded in the thylakoid membranes of chloroplasts. Each mole of 680 nm photons provides approximately \(176\,\mathrm{kJ}\) of energy.

\vspace{0.3em}
\noindent\textit{Net reaction:}
\begin{align}
2\,\mathrm{H}_2\mathrm{O} 
&+ 2\,\mathrm{NADP}^+ 
+ 3\,\mathrm{ADP} 
+ 3\,\mathrm{P}_i 
+ h\nu \nonumber \\
&\rightarrow 2\,\mathrm{NADPH} 
+ 3\,\mathrm{ATP} 
+ \mathrm{O}_2.
\end{align}

ATP and NADPH are consumed in the fixation of carbon. The quantum requirement is roughly 8–10 photons per molecule of \(\mathrm{CO}_2\) fixed.

\noindent\textbf{Carbon Fixation and Biomass Synthesis}\\[0.5em]
In the Calvin–Benson cycle, carbon dioxide is enzymatically fixed into triose phosphates using the energy carriers from the light reactions. The overall reaction for one glucose unit is:

\begin{align}
6\,\mathrm{CO}_2 
&+ 18\,\mathrm{ATP} 
+ 12\,\mathrm{NADPH} \nonumber \\
&\rightarrow \mathrm{C}_6\mathrm{H}_{12}\mathrm{O}_6 
+ 18\,\mathrm{ADP} 
+ 18\,\mathrm{P}_i 
+ 12\,\mathrm{NADP}^+.
\end{align}

Glucose is polymerized into cellulose by dehydration:

\begin{align}
n\,\mathrm{C}_6\mathrm{H}_{12}\mathrm{O}_6 
&\rightarrow (\mathrm{C}_6\mathrm{H}_{10}\mathrm{O}_5)_n 
+ n\,\mathrm{H}_2\mathrm{O}.
\end{align}

These polymers form the primary structure of wood (secondary xylem), alongside lignin and hemicellulose.

\noindent\textbf{Time-Integrated Mass Gain}\\[0.5em]
Assuming an annual net primary productivity (NPP) of \(10^4\,\mathrm{kg/ha}\) of dry biomass, with 50\% carbon content:

\vspace{0.3em}
\noindent\textit{Carbon fixed per year:}
\begin{align}
5 \times 10^6\,\mathrm{g\,C/ha} 
&= \frac{5 \times 10^6}{12}\,\mathrm{mol\,C/ha} \nonumber \\
&= 4.17 \times 10^5\,\mathrm{mol}, \\
4.17 \times 10^5\,\mathrm{mol\,CO}_2 
&\times 44\,\mathrm{g/mol} 
= 1.84 \times 10^7\,\mathrm{g\,CO}_2 \\
&= 18.4\,\mathrm{tonnes\,CO}_2/\mathrm{ha}/\mathrm{year}.
\end{align}

\noindent\textbf{Tree-Scale Time Extrapolation}\\[0.5em]
Assuming 100 trees per hectare and uniform distribution, each tree fixes:

\begin{align}
\frac{18.4\,\mathrm{tonnes\,CO}_2}{100} 
&= 184\,\mathrm{kg\,CO}_2/\text{tree/year}, \\
184\,\mathrm{kg/year} \times 50 
&= 9.2\,\mathrm{tonnes\,CO}_2/\text{tree}, \\
9.2 \times 0.27 
&= 2.48\,\mathrm{tonnes\,C/tree}.
\end{align}

This corresponds to approximately 5 tonnes of total dry biomass per tree — consistent with measured values for mature broadleaf species in temperate forests.

\noindent\textbf{Elemental Mass Contribution}\\[0.5em]
The typical dry mass composition of tree tissue is:

\begin{itemize}[leftmargin=*]
  \item Carbon: 45–50\% (from atmospheric \(\mathrm{CO}_2\))
  \item Oxygen: 40–45\% (~\textbf{⅔ from \(\mathrm{CO}_2\)}, ~\textbf{⅓ from \(\mathrm{H}_2\mathrm{O}\)})
  \item Hydrogen: ~6\% (from water)
  \item Minerals: 1–5\% (from soil: N, P, K, Ca, etc.)
\end{itemize}

These atoms are assembled into ordered biopolymers with high mechanical stability and long-term carbon storage.

\vspace{0.5em}
\noindent\textbf{References:}\\
Farquhar, G. D., von Caemmerer, S. (1980). \textit{Planta}, \textbf{149}, 78–90.\\
Taiz, L., Zeiger, E. (2010). \textit{Plant Physiology}.
\end{technical}
