\begin{historical}
In 1895, Ludwig Boltzmann proposed that the observed low entropy of the universe might arise as a rare fluctuation within a vastly larger equilibrium state. His goal was to reconcile the second law of thermodynamics with the possibility of eternal time: if the universe is statistically dominated by high-entropy configurations, then any low-entropy region — such as our observable cosmos — would have to be an exceptional, temporary departure.

This explanation, however, faced immediate conceptual challenges. Boltzmann's contemporaries pointed out that smaller fluctuations are exponentially more probable than large ones. Hence, it would be statistically more likely for a single observer, complete with illusory memories, to emerge briefly from equilibrium than for an entire universe to evolve coherently from low entropy.

The idea re-emerged in the late 20th century as the "Boltzmann Brain" problem — an absurd consequence of taking certain cosmological models literally. If the universe lasts long enough, and if thermal or quantum fluctuations occur eternally, then most "observers" should be momentary self-aware configurations with fabricated pasts. In this context, Boltzmann’s original idea became a cautionary tale about reasoning backward from present experience in a probabilistic universe.

Similar logic has appeared outside physics. A strain of religious apologetics proposes that fossils, rock strata, and other signs of age might have been created fully formed. This bypasses historical causality in favor of constructed appearance — an outlook structurally similar to the Boltzmann Brain concept, though motivated by theological rather than thermodynamic reasoning.
\end{historical}
