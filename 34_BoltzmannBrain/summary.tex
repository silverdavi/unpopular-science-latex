The Boltzmann Brain paradox emerges from applying statistical physics to cosmology: in infinite time, random fluctuations in a high-entropy universe should produce isolated conscious entities far more frequently than entire ordered universes like ours. Mathematical analysis suggests a single brain with illusory memories requires vastly fewer unlikely coincidences than a genuine cosmic history. These hypothetical self-aware systems would possess false memories of existing in a structured world, despite being momentary arrangements with no causal history. The paradox creates cognitive instability — if observers are statistical fluctuations with false memories rather than evolved beings, the reliability of cosmological theories themselves becomes suspect.
