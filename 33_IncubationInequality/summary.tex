The Gaussian Correlation Inequality states that for two convex, symmetric sets in high-dimensional space, the probability of a Gaussian-distributed random point landing in both regions simultaneously is at least the product of the individual probabilities. This relationship, connecting geometric structure to probabilistic behavior, remained unproven for over fifty years despite verification in special cases. In 2014, Thomas Royen, a retired statistics professor from a university of applied sciences, resolved the conjecture using elementary tools: Laplace transforms and properties of gamma distributions.
