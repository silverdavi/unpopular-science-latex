\fullpageexercises[The Universality of Linearity of Expectation]
{
\section*{Riddles Solved by Linearity of Expectation}

\noindent
Linearity of expectation states that for any finite collection of random variables \( X_1, \dots, X_n \), regardless of dependence,
\[
\mathbb{E}\left[\sum_{i=1}^n X_i\right] = \sum_{i=1}^n \mathbb{E}[X_i].
\]
This principle applies even when the variables interact in complex or unknown ways. The following riddles highlight its surprising power: each involves a probabilistic setting with hidden structure, yet a clean result follows from applying linearity directly—without requiring independence, variance, or even probability mass functions.

\vspace{1em}
\begin{enumerate}

    \item \textbf{Red Sphere and Cube.}\\
    The surface of a sphere is painted red and blue, with 90\% of the area red. A cube is placed inside the sphere such that all eight vertices touch the surface.\\
    Prove that there exists an orientation where all 8 vertices lie on red.\\[0.5em]
    \textit{Hint:} What happens if you average over all possible cube rotations?

    \item \textbf{Coins Covering Arbitrary Points.}\\
    Show that any 10 points in the plane can be simultaneously covered by 10 non-overlapping unit coins.\\[0.5em]
    \textit{Hint:} Think spatially: what would happen if a dense coin tiling were randomly shifted?

    \item \textbf{Empty Bins.}\\
    Throw \( n \) balls uniformly at random into \( n \) bins. What is the expected number of empty bins?\\[0.5em]
    \textit{Hint:} Define indicator variables for each bin being empty, then evaluate their expectation.

    \item \textbf{Coupon Collector.}\\
    What is the expected number of samples required to collect all \( n \) types of coupons, where each draw returns a uniformly random coupon?\\[0.5em]
    \textit{Hint:} Consider the waiting time to acquire each new coupon, one at a time.

    \item \textbf{Permutation Fixed Points.}\\
    Let \( \sigma \) be a uniformly random permutation of \( n \) elements. What is the expected number of fixed points?\\[0.5em]
    \textit{Hint:} Think in terms of expected value per element, rather than analyzing cycles.

    \item \textbf{Two Beverages per Villager.}\\
    Each villager drinks exactly two out of three beverages: tea, beer, or wine. Suppose that 1/2 of villagers drink tea, 2/3 drink beer, and 3/4 drink wine. Prove that this situation is impossible.\\[0.5em]
    \textit{Hint:} Add up the expected beverage counts across the population.

    \item \textbf{Expected Maximum Red Degree.}\\
    100 people are connected pairwise by either a red or blue edge. Show that some person has at least 49 red connections.\\[0.5em]
    \textit{Hint:} What does the average red degree tell you about the maximum?

\end{enumerate}

\noindent\hrulefill

\begin{center}
\rotatebox[origin=c]{180}{%
\begin{minipage}{0.95\textwidth}
\small
\noindent
\(
\textbf{Answers: }\ 
\text{1: } \mathbb{E}[\text{\# red vertices}] = 7.2 \Rightarrow \text{some rotation achieves 8} \quad
\text{2: } \text{expected coverage } > 9 \Rightarrow \text{some shift covers all 10} \quad
\text{3: } \text{Each bin empty w.p. } (1 - 1/n)^n \Rightarrow \mathbb{E} \approx n/e \quad
\text{4: } \mathbb{E} = n \sum_{k=1}^n \frac{1}{k} \sim n \log n \quad
\text{5: } \mathbb{E} = \sum_{i=1}^n \frac{1}{n} = 1 \quad
\text{6: } \mathbb{E}[N] = \frac{1}{2} + \frac{2}{3} + \frac{3}{4} = \frac{23}{12} > 2 \quad
\text{7: } \text{Average degree } > 48 \Rightarrow \text{some node } \ge 49.
\)
\end{minipage}
}
\end{center}
}
