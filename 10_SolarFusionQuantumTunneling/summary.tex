Solar fusion proceeds despite temperatures insufficient for classical nuclear reactions because quantum tunneling enables protons to penetrate the Coulomb barrier with non-zero probability. At the Sun's core temperature of 15 million Kelvin, the average proton possesses only about 1/20 the energy classically required to overcome electromagnetic repulsion between positively charged nuclei. Quantum mechanics allows particles to "tunnel" through energy barriers they cannot surmount classically, with probability decreasing exponentially with barrier height and width. This tunneling effect, combined with the enormous number of interaction attempts in the solar plasma, sustains the precise fusion rate necessary for stellar stability over billions of years.
