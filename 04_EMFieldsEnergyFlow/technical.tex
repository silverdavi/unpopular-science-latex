\begin{technical}
% Add at the beginning of the file
\sloppy
{\Large\textbf{Electromagnetic Energy Flow Outside Wires}}

\noindent\textbf{Fields, Not Wires, Carry the Energy (Maxwell \& Poynting)}\\[0.5em]
The common view that current travels inside a conductor to power devices is incomplete. Maxwell’s equations reveal that \textit{electromagnetic fields}, rather than moving electrons, transport energy. In free space, Maxwell’s equations in SI units are:
\begin{align*}
\nabla \cdot \mathbf{E} &= \frac{\rho}{\varepsilon_0}, 
&\quad \nabla \cdot \mathbf{B} &= 0,\\
\nabla \times \mathbf{E} &= -\frac{\partial \mathbf{B}}{\partial t},
&\quad \nabla \times \mathbf{B} &= \mu_0 \mathbf{J} + \mu_0 \varepsilon_0 \frac{\partial \mathbf{E}}{\partial t},
\end{align*}
where \(\mathbf{E}\) and \(\mathbf{B}\) are the electric and magnetic fields, \(\rho\) is charge density, and \(\mathbf{J}\) is current density. The \textit{Poynting vector}, defined as
\[
\mathbf{S} = \frac{1}{\mu_0} (\mathbf{E} \times \mathbf{B}),
\]
describes the direction and magnitude of energy flow. In a typical circuit, \(\mathbf{S}\) is concentrated in the space around conductors, not within them, showing that fields, not electron drift, convey energy.

\vspace{0.7em}
\noindent\textbf{Near a Conductor: Shaping the Fields}\\[0.5em]
Wires carry charges that generate \(\mathbf{E}\) and \(\mathbf{B}\), but the power flux \(\mathbf{S}\) remains predominantly outside. The relationship is structured as follows:
\begin{itemize}[leftmargin=*]
\item The applied \textit{voltage} establishes the electric field \(\mathbf{E}\). 
\item The \textit{current} generates the magnetic field \(\mathbf{B}\).
\item The cross product \(\mathbf{E} \times \mathbf{B}\) directs energy flow outside the conductor.
\item Conductors constrain and guide the fields, enabling controlled power transfer with minimal radiation losses.
\end{itemize}
While conduction electrons move within wires at a slow drift velocity, energy transmission occurs in the surrounding field region.

\vspace{0.7em}
\noindent\textbf{Electrons Are Slow}\\[0.5em]
The speed of electrons in a wire, known as the \textit{drift velocity}, is given by:
\[
v_d = \frac{I}{n q A},
\]
where \( I \) is the current, \( n \) is the number density of free electrons in the conductor, \( q \) is the elementary charge, and \( A \) is the cross-sectional area of the wire.

\noindent\textbf{Example.} For a copper wire of cross-sectional area \( A = 1 \, \text{mm}^2 = 1 \times 10^{-6} \, \text{m}^2 \), carrying a current of \( I = 3 \, \text{A} \), and using:
\[
n \approx 8.5 \times 10^{28} \, \text{electrons/m}^3, \quad q = 1.6 \times 10^{-19} \, \text{C},
\]
the drift velocity is:
\[
v_d \approx \frac{3}{(8 \cdot 10^{28})(10^{-19})(10^{-6})} \approx 2.2 \cdot 10^{-4} \, \text{m/s}.
\]
That is, electrons drift at roughly 0.2 mm per second.

\vspace{1em}
\noindent\textbf{Bonus Section: Maxwell’s Equations in 4D Differential Forms (Warning: Jargon Ahead)}\\[0.5em]
A remarkably elegant formulation
uses differential forms in four-dimensional spacetime. Instead of treating electric and magnetic fields separately, one defines the field-strength 2-form \(F\) from a potential 1-form \(A\):  
\[
F = \mathrm{d}A.
\]
Maxwell’s equations in vacuum then reduce to:
\[
\mathrm{d}F = 0, 
\quad
\mathrm{d}(*F) = \mu_0 J,
\]
where \(*F\) is the Hodge dual of \(F\), and \(J\) is the 1-form representing charge and current. These equations encapsulate:
\begin{itemize}[leftmargin=*]
\item \(\mathrm{d}F = 0\): Magnetic fields are divergence-free (no monopoles), and electric fields induce magnetic circulation.
\item \(\mathrm{d}(*F) = \mu_0 J\): Charge and current generate fields, unifying Gauss’s law and Ampère’s law.
\end{itemize}
This formulation emphasizes that electromagnetic phenomena, including the Poynting vector, arise naturally from spacetime geometry rather than as separate electric and magnetic field concepts in three-dimensional space. 

While defining these objects and proving their properties can take months, the payoff is nice: results like Maxwell’s equations emerge from simple geometric principles. Other results, such as the generalized Stokes’ theorem, follow with similar elegance.

\vspace{0.7em}
\noindent\textbf{References}\\
Feynman, R. P., Leighton, R. B., \& Sands, M. (1964). \textit{The Feynman Lectures on Physics, Vol. II}. Addison-Wesley.\\
Baez, J. C., \& Muniain, J. P. (1994). \textit{Gauge Fields, Knots and Gravity}. Series on Knots and Everything, vol. 4, World Scientific Publishing.\\
\end{technical}