Electrical energy travels primarily through electromagnetic fields surrounding conductors, not through the movement of electrons in wires. While electrons drift at millimeters per second, energy transfer occurs near light speed through the Poynting vector (S = E × B), which describes energy flow perpendicular to both electric and magnetic fields. This field-based transmission explains why circuits respond almost instantly despite slow electron movement, challenging the common misconception that electricity flows like water through pipes. Rather, the phenomenon resembles waves propagating across water's surface, where energy travels through the medium while the particles themselves undergo minimal displacement.
