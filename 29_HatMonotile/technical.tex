\begin{technical}
{\Large\textbf{The Spectre and the Geometry of Chiral Aperiodic Tilings}}\\[0.7em]

\textbf{Introduction}\\[0.5em]
This section formalizes the conditions under which the hat and Spectre tiles enforce aperiodicity in the Euclidean plane. Aperiodic tilings cover the plane without gaps or overlaps, yet admit no translational symmetries. The hat tile is a polygonal monotile whose tilings require both itself and its mirror image. The Spectre, derived from an equilateral variant of the hat, enforces aperiodicity even under homochiral constraints — using only orientation-preserving isometries. Its aperiodicity arises from hierarchical substitution structure, emergent local matching constraints, and a provable failure of global periodicity.

\textbf{Tiling Definitions and Aperiodicity}\\[0.5em]
Let \( T \subset \mathbb{E}^2 \) be a compact, connected region (a closed topological disk). A \emph{monohedral tiling} is a countable collection \( \mathcal{T} = \{g_i T\}_{i \in \mathbb{N}} \), where each \( g_i \) is an isometry of the Euclidean plane, such that the tiles have disjoint interiors and their union is the entire plane. A tiling \( \mathcal{T} \) is \emph{periodic} if there exists a nonzero translation vector \( \mathbf{v} \in \mathbb{R}^2 \) such that \( \mathcal{T} + \mathbf{v} = \mathcal{T} \). A tile \( T \) is \emph{aperiodic} if every monohedral tiling it admits is non-periodic.

\textbf{Chirality and Classification of Monotiles}\\[0.5em]
An isometry is orientation-preserving if it consists of translations or rotations; orientation-reversing isometries include reflections. A tiling is called \emph{homochiral} if all tiles are related by orientation-preserving isometries. A monotile \( T \) is a:
\begin{itemize}
\item \emph{weakly chiral aperiodic monotile} if all of its homochiral tilings are aperiodic,
\item \emph{strictly chiral aperiodic monotile} if it admits only homochiral tilings and all of them are aperiodic.
\end{itemize}
The hat tile (a union of eight right kites) is aperiodic when both \( T \) and its mirror image \( \bar{T} \) are permitted. Its equilateral counterpart, Tile(1,1), becomes strictly chiral when modified to yield the Spectre — whose edge geometry excludes reflection-based tilings by construction.

\textbf{Substitution Systems and Hierarchical Nesting}\\[0.5em]
To prove aperiodicity, we use \emph{substitution tilings}: constructions where every tile lies within an infinite nested sequence of larger clusters, or supertiles. A tile set \( \mathcal{A} \) is hierarchical if each tiling admitted by \( \mathcal{A} \) admits a unique infinite hierarchy of supertiles under a fixed substitution rule \( \sigma \). Formally, let
\[
\sigma^n(T) = \text{level-}n \text{ supertile}, \quad \text{with } \sigma^n(T) \subset \sigma^{n+1}(T).
\]
Uniqueness implies non-periodicity: if a tiling were periodic, some large supertile would intersect a translated copy of itself, violating the rigidity of the hierarchy. The Spectre admits two inflation rules:
\begin{enumerate}
    \item A Spectre becomes a cluster of seven Spectres and one Mystic (a symmetric two-tile compound),
    \item A Mystic becomes a cluster of six Spectres and another Mystic.
\end{enumerate}
These rules alternate handedness at each level. The number of tiles and their distribution grow by the Perron–Frobenius eigenvalue \( \lambda = 4 + \sqrt{15} \), an irrational factor that excludes periodic repeatability.

\textbf{Combinatorial Equivalence and Reduction to Hexagons}\\[0.5em]
A key idea in the proof is \emph{combinatorial equivalence}: two tilings are equivalent if their incidence complexes (vertices, edges, tile adjacencies) are topologically identical. Every Spectre tiling is equivalent to a tiling by hats and turtles aligned to a [3.4.6.4] kite-based grid. These in turn are grouped into bounded clusters (e.g., one turtle surrounded by seven or eight hats), which form units called T7H and T8H. These clusters can be replaced by nine \emph{marked hexagons}, each labeled with edge constraints that enforce how tiles can meet.

Let \( \mathcal{H} = \{H_1, \dots, H_9\} \) be the set of these marked hexagons. They admit a deterministic substitution structure:
\begin{align}
H_i \mapsto \text{cluster of } H_j \in \mathcal{H}, \\\quad \text{with unique nesting at each level}.
\end{align}
This system is recursive, rigid, and non-overlapping. Since the substitution rules imply a unique infinite hierarchy for any tiling, and the inflation factor is irrational, the marked hexagon system — and therefore the Spectre — admits only non-periodic tilings.

\vspace{0.5em}
\textbf{References:}\\
Smith, D., Myers, J.\,S., Kaplan, C.\,S., \& Goodman-Strauss, C. (2024). \emph{A chiral aperiodic monotile}. \textit{Combinatorial Theory}, 4(2), \#13.\\
Goodman-Strauss, C. (1999). \emph{Aperiodic hierarchical tilings}. In: \textit{Foams and Emulsions}, NATO ASI Series E, vol. 354, pp. 481–496.
\end{technical}
