\begin{technical}
{\Large\textbf{Resolving the Envelope Paradox}}\\[0.7em]

\textbf{Introduction}\\[0.5em]
The paradox emerges from treating the observed amount $Y$ as equally likely to be the smaller or larger of two amounts related by a $1:2$ ratio. The naive expectation then assumes:
\[
\mathbb{E}[\text{switch}] = 0.5 \cdot 2Y + 0.5 \cdot \frac{Y}{2} = \frac{5}{4}Y,
\]
suggesting a gain from switching. But this treats $Y$ as both $X$ and $2X$ in different terms of the sum — a misapplication of expectation under conditioning.

\textbf{Correct Conditioning and Model Specification}\\[0.5em]
Let $X$ be a positive random variable drawn from a prior distribution $f(x)$. The envelope pair $(X, 2X)$ is constructed, and one envelope is presented at random. Let $Y$ denote the observed amount. There are two possibilities:
\begin{align}
Y = X &\Rightarrow \text{other envelope has } 2Y,\\
Y = 2X &\Rightarrow \text{other envelope has } \frac{Y}{2}.
\end{align}
These correspond to disjoint events in the prior:
\begin{align}
\mathbb{P}(X = Y) &\propto f(Y), \\
\mathbb{P}(X = Y/2) &\propto f(Y/2).
\end{align}
The expected value of switching given $Y$ is:
\begin{align}
\mathbb{E}[\text{switch} \mid Y] 
= \frac{2Y \cdot f(Y) + \tfrac{1}{2}Y \cdot f(Y/2)}{f(Y) + f(Y/2)}.
\end{align}
This ratio depends entirely on the shape of $f$. When $f$ decays rapidly, $f(Y/2) \gg f(Y)$ for large $Y$, implying $Y$ is more likely to be the larger value and switching is unfavorable.

\textbf{Example: Pareto Prior}\\[0.5em]
Let $X \sim \text{Pareto}(\alpha)$ on $[1, \infty)$:
\[
f(x) = \alpha x^{-(\alpha + 1)}, \quad x \ge 1.
\]
Then for $Y \ge 2$, we compute:
\begin{align}
f(Y) &= \alpha Y^{-(\alpha + 1)}, \\
f(Y/2) &= \alpha (Y/2)^{-(\alpha + 1)} = \alpha \cdot 2^{\alpha + 1} \cdot Y^{-(\alpha + 1)}.
\end{align}
Thus:
\[
\mathbb{E}[\text{switch} \mid Y] = \frac{2Y + \tfrac{1}{2}Y \cdot 2^{\alpha + 1}}{1 + 2^{\alpha + 1}} = Y \cdot \frac{2 + 2^{\alpha}}{1 + 2^{\alpha + 1}}.
\]
This expression may be above or below $Y$ depending on $\alpha$.

\textbf{Improper Priors and Divergence}\\[0.5em]
Now suppose $X$ is drawn from a log-uniform distribution: $f(x) \propto 1/x$, over $[1, \infty)$. This is an improper prior — it cannot be normalized:
\[
\int_1^{\infty} \frac{1}{x} dx = \infty.
\]
Nonetheless, applying the same density ratio logic leads to:
\begin{align}
f(Y) &= \frac{1}{Y}, \quad f(Y/2) = \frac{2}{Y},\\
\mathbb{E}[\text{switch} \mid Y] &= \frac{2Y \cdot \frac{1}{Y} + \tfrac{1}{2}Y \cdot \frac{2}{Y}}{\frac{1}{Y} + \frac{2}{Y}} = \frac{2 + 1}{3}Y = Y.
\end{align}
Here, the expectation equals $Y$, suggesting switching is neutral. But this outcome is misleading: since the prior is improper, the marginal distribution over $Y$ is undefined, and so is the overall expected value. 

\textbf{Finite Uniform Model}\\[0.5em]
Let the smaller amount $x$ be drawn uniformly from the set $\{2^0, 2^1, \dots, 2^{N-1}\}$. Each pair $(x, 2x)$ is equally likely, and one of the two values is observed. Let $A = 2^m$ be the observed amount.

For interior values $1 \le m \le N - 2$, both interpretations $A = x$ and $A = 2x$ are possible, yielding an expected gain:
\[
\mathbb{E}[\Delta \mid A = 2^m] = \tfrac{1}{2}(2^m) + \tfrac{1}{2}(-2^{m-1}) = 2^{m-2}.
\]
At the boundaries, only one interpretation is valid:
\[
\mathbb{E}[\Delta \mid A = 2^0] = +1, \quad
\mathbb{E}[\Delta \mid A = 2^{N-1}] = -2^{N-2}.
\]
Averaging over all $N$ possible values yields the global expectation:
\begin{align}
\mathbb{E}[\Delta] &= \frac{1}{N} \left( \sum_{m=1}^{N-2} 2^{m-2} + 1 - 2^{N-2} \right)\\
&= \frac{1}{N} \left( -2^{N-3} + \tfrac{1}{2} \right).\\
&< 0
\end{align}
As $N \to \infty$, this approaches zero from below, confirming that switching has no long-run advantage once edge effects are accounted for.

\vspace{0.5em}
\textbf{References:}\\
Nalebuff, B. (1989). The Other Person’s Envelope Is Always Greener. \textit{J. Econ. Persp.}, \textbf{3}(1), 171--181.
\end{technical}
