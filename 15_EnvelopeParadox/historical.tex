\begin{historical}
The mathematical formalization of probability began in earnest during the 18th century, driven by practical problems in gambling and finance. Jacob Bernoulli’s \textit{Ars Conjectandi} (1713) laid the groundwork for the concept of expected value, proposing that uncertain outcomes could be treated with numerical precision. Abraham de Moivre expanded this framework in \textit{The Doctrine of Chances} (1718), introducing systematic techniques for computing probabilities in complex games of chance.

By the 19th century, these ideas had matured into a formal discipline, yet they still struggled with edge cases and paradoxes that defied intuitive reasoning. One such class of problems involved scenarios with infinite or unbounded outcomes, where naive use of expected value produced misleading conclusions. The St. Petersburg Paradox, analyzed by Daniel Bernoulli in 1738, was among the first to highlight the pitfalls of applying expectation without constraints on utility or distribution. 

The Envelope Paradox emerged much later, but it inherits this lineage. In the late 20th century, a renewed interest in decision theory and Bayesian analysis brought the paradox to prominence. Philosophers and mathematicians, including Raymond Smullyan and Clark Glymour, used it to probe how information updates and hidden assumptions affect rational choice.
\end{historical}
