\begin{historical}
The notion that phases of matter could be classified by global rather than local features first emerged from the study of the quantum Hall effect — a phenomenon in which a transverse voltage develops across a two-dimensional electron system under strong magnetic fields, leading to quantized this Hall conductance. In 1980, Klaus von Klitzing discovered that in such systems, Hall conductance appears in exact integer steps. This quantization resisted explanation within traditional symmetry-based frameworks. Two years later, Thouless, Kohmoto, Nightingale, and den Nijs proposed that these steps correspond to topological invariants defined over momentum space — a connection that introduced topology into condensed matter theory in a mathematically precise way.

Throughout the 1990s, interest grew in states of matter that could not be described by broken symmetry or conventional order parameters. Theoretical work on topological order and fractional statistics suggested that global wavefunction structure could underlie distinct quantum phases. These ideas remained largely decoupled from band theory until the mid-2000s, when Charles Kane and Eugene Mele introduced a model for spin-orbit coupled electrons in graphene. Their prediction of a quantum spin Hall effect — with dissipationless edge modes protected by time-reversal symmetry — offered a new phase of matter in which topology governed the behavior of ordinary band electrons.

In 2007, Laurens Molenkamp’s group experimentally observed quantized edge transport in HgTe/CdTe quantum wells — layered heterostructures composed of mercury telluride and cadmium telluride — confirming the quantum spin Hall effect and marking the first realization of a two-dimensional topological insulator. Subsequent theoretical work by Fu, Kane, and Mele, and independently by Moore and Balents, extended the classification to three dimensions. They predicted that certain narrow-gap semiconductors, including Bi\(_2\)Se\(_3\) and Bi\(_2\)Te\(_3\), would exhibit conducting surface states protected by time-reversal symmetry and classified by a \(\mathbb{Z}_2\) index. The \(\mathbb{Z}_2\) index distinguishes between two types of band structures — those that are topologically trivial and those that necessarily host robust, symmetry-protected surface states.

These predictions were confirmed in 2008–2009 through angle-resolved photoemission spectroscopy (ARPES), which resolved the gap-spanning surface bands in these materials. What began as a refinement of quantum Hall physics had by then evolved into a new kind of phase classification — one grounded in the global topology of quantum states.
\end{historical}
