\begin{technical}
{\Large\textbf{Radon, Hough, and the Geometry of Detection}}\\[0.7em]

\textbf{Introduction}\\[0.5em]
The detection of geometric structures in an image may be posed as the identification of parameter values $p$ for which a corresponding shape $c(p)$ is present in the data. Both the Radon and Hough transforms implement this principle — one via continuous integration, the other through discrete accumulation. Despite operational differences, both realize the same mathematical mapping: from image domain $\mathbb{R}^2$ to a parameter manifold $P$, governed by a constraint kernel of the form $\delta(C(x; p))$. Their distinction is epistemological — the manner of how data is interrogated — but not ontological.

\textbf{Transform Duality: Reading vs Writing}\\[0.5em]
Let $I(x)$ be a spatial image and $p \in P$ a parameter vector. The Radon transform is defined as the integral projection:
\[
R(p) = \int_{\mathbb{R}^n} I(x)\, \delta(C(x; p))\, dx.
\]
This is a \emph{reading} operation: for each $p$, it queries all $x$ that satisfy $C(x; p) = 0$ and accumulates their intensity. In contrast, the Hough transform performs a \emph{writing} operation. For each $x_0$ such that $I(x_0) \ne 0$, it computes all $p$ for which $C(x_0; p) = 0$ and increments $H(p)$.

Let $\mathcal{C}_p = \{x \in \mathbb{R}^n \mid C(x; p) = 0\}$ denote the pre-image of shape $p$, and $\mathcal{M}_x = \{p \in P \mid C(x; p) = 0\}$ the image of feature $x$. Then:
\begin{align*}
R(p) &= \int_{\mathcal{C}_p} I(x)\, d\mu(x), \quad \text{(Radon)}\\
H(p) &= \sum_{x \in \mathrm{supp}(I)} \mathbf{1}_{\mathcal{M}_x}(p). \quad \text{(Hough)}
\end{align*}
The former computes the inner product between $I(x)$ and a template over $\mathcal{C}_p$, while the latter constructs a discrete indicator function supported on a union of manifolds.

\textbf{Generalized Template Algebra}\\[0.5em]
Let $C(p,x)$ be a generalized function in the sense of Gel'fand, with $C(p,x) = \delta(C(x; p))$. Then the transform is a Fredholm integral operator:
\[
(\mathcal{L}_C I)(p) = \int_{\mathbb{R}^n} C(p,x)\, I(x)\, dx.
\]
This formulation unifies both transforms with classical template matching. If $C(p,x)$ is shift-invariant, i.e., $C(p,x) = K(x - \phi(p))$, the operator reduces to convolution. This links Radon/Hough methods to Fourier-analytic descriptors.

In $\mathbb{R}^2$, the detection of lines in normal form $\rho = x\cos\theta + y\sin\theta$ yields
\[
C(\rho,\theta; x,y) = \delta(\rho - x\cos\theta - y\sin\theta).
\]
Substituting into $\mathcal{L}_C I$ recovers the Radon projection; replacing the integral with discrete summation recovers the Hough vote.

\textbf{Intersection Topology in Parameter Space}\\[0.5em]
Every edge point $x$ defines a manifold $\mathcal{M}_x \subset P$. True structures arise where multiple $\mathcal{M}_x$ intersect. The locus $\bigcap_{x \in S} \mathcal{M}_x$ defines candidate parameters supported by the feature set $S$. This is a geometric analogue of data coherence. If the intersection has nonzero measure, a shape is present. Noise and discretization perturb $\mathcal{M}_x$, but if the intersections persist, the structure is robust.

In the continuous limit, $H(p) \to R(p)$ as $I(x) \to \delta$ on features and as the sampling density increases. Thus, the Hough transform is a discrete approximation to the Radon transform under sparse feature distributions.

\textbf{Transform-Invariant Detection and Group Actions}\\[0.5em]
Let $\mathcal{G}$ be a Lie group acting on image space via $\phi_g: \mathbb{R}^n \to \mathbb{R}^n$. A template $T$ transforms covariantly if
\[
C(p, \phi_g(x)) = C(g^{-1} \cdot p, x).
\]
This defines equivariant detection: shape presence is preserved under group action. Both the Radon and Hough transforms naturally extend to such scenarios. For example, detecting rotated ellipses or scaled spirals requires only reparameterizing the kernel over the associated group orbit.

\vspace{0.5em}
\textbf{References:}\\[0.5em]
M. van Ginkel, C. L. Luengo Hendriks, and L. J. van Vliet (2004). \textit{A short introduction to the Radon and Hough transforms and how they relate to each other}. TU Delft Technical Report QI-2004-01.\\
Gel’fand, I. M., \& Shilov, G. E. (1964). \textit{Generalized Functions, Vol. I: Properties and Operations}. Academic Press.
\end{technical}
