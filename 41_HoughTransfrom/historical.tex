
\begin{historical}
In 1962, Paul V. C. Hough filed a patent for a method of identifying complex patterns in visual data by mapping image features into a parameter space. His technique, conceived during the early era of digital imaging, converted edge points in a photographic plane into sinusoids in a polar-coordinate domain. This reformulation allowed collinear features to be detected as intersecting curves in the transformed space, circumventing the combinatorial burden of brute-force pixel-space alignment.

Early implementations relied on analog hardware and optical computing elements. Limitations in memory and processing speed constrained accumulator resolution and the range of detectable geometries. Despite these constraints, the method was adopted in early automation systems for detecting weld lines, highway markings, and parts in mechanical assemblies.

In 1972, Richard Duda and Peter Hart provided the first formal exposition of the method in their landmark paper, reframing it as a discrete voting process over a bounded parameter space. Their formulation gave the method the name "Hough transform" and connected it to broader principles in statistical decision theory. Subsequent extensions by Ballard and others generalized the idea to arbitrary curves and spatial templates, enabling detection of circles, ellipses, and parabolas.

By the 1980s and 1990s, with advances in digital signal processors and parallel computing, the Hough transform became a foundational tool in industrial vision systems, robotics, and medical image reconstruction. Its structure-preserving mapping from image to parameter space allowed for robust detection even in the presence of occlusion, fragmentation, and noise  —  a capacity that remains central to modern implementations in lane detection, tomography, and motion analysis.
\end{historical}