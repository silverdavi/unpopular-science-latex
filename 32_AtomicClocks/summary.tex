Timekeeping has progressively moved toward smaller physical phenomena: from Earth's rotation to pendulums, from crystal oscillations to atomic transitions, and now toward nuclear resonances. The SI second, defined by 9,192,631,770 periods of cesium-133's hyperfine transition, relies on quantum interactions between nuclear and electronic magnetic moments. This shift to microscopic reference standards improves precision exponentially — hydrogen masers achieve stability through atomic transitions at 1.42 GHz, while optical lattice clocks using strontium reach fractional uncertainties below 10⁻¹⁸ by probing transitions at ~10¹⁵ Hz. The progression continues toward nuclear clocks using thorium-229, which promises quality factors above 10¹⁹ by exploiting transitions in atomic nuclei rather than electron shells.