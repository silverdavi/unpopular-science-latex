Poisson's spot (also called the Arago spot) demonstrates wave diffraction through the unexpected appearance of a bright point at the center of a circular object's shadow. When Augustin-Jean Fresnel proposed light as a wave phenomenon in 1818, Siméon Poisson derived this counterintuitive prediction to disprove the theory. According to wave principles, light diffracting around a circular obstacle creates constructive interference exactly at the center point where waves from all directions arrive in phase. François Arago's experimental confirmation of this bright central spot, which had seemed absurd under particle theory, provided compelling evidence for the wave nature of light, transforming what was intended as a refutation into definitive validation.
