\begin{historical}
At the start of the 19th century, France’s scientific institutions were still dominated by \textbf{Isaac Newton}’s \emph{corpuscular theory} of light (the light is made of small particles). Although Newton was British, his mechanical worldview had become the orthodox framework for explaining optical phenomena in France. This dominance was reinforced not only by Newton’s authority but by institutional inertia within the \emph{Académie des Sciences}, where many senior members — including influential mathematicians like \textbf{Siméon Denis Poisson} — had built careers aligned with corpuscular assumptions.

In contrast, \textbf{Christiaan Huygens}’ earlier \emph{wave theory}, though developed in Paris in the late 17th century, had fallen out of favor. His principle — that every point on a wavefront acts as a source of secondary wavelets — was largely viewed as a heuristic, lacking the mechanical precision demanded by the Newtonian establishment.

The political context mattered. In the wake of the Napoleonic Wars, French science was undergoing institutional consolidation. State-sponsored prizes and commissions were used not only to reward discovery but to regulate the boundaries of accepted science. The Académie’s \emph{Grand Prix} competitions served both purposes, offering public recognition while reinforcing prevailing views. When the 1818 competition on diffraction was announced, it was not simply a call for explanation — it was a test of theoretical allegiance.

Into this charged environment entered \textbf{Augustin-Jean Fresnel}, a provincial engineer with no formal academic post. He submitted a comprehensive wave-based theory of light that relied on interference and diffraction as foundational, not anomalous, behaviors. Fresnel’s mathematics extended Huygens’ ideas, introducing integrals and phase relations to predict light intensity patterns. For many at the Academy, this framework was intellectually foreign and politically unwelcome.

Poisson, serving on the jury, was representative of the old guard: a formalist, a Newtonian, and a skeptic of wave-based optics. \textbf{Dominique-François Arago}, also on the committee, occupied a more ambiguous position. Though not yet fully aligned with the wave camp, Arago had corresponded with Fresnel and was known for his openness to alternative frameworks. The committee thus represented an ideological fault line within French science: Newtonian orthodoxy versus a nascent wave revival. What followed would be a confrontation not only of theories but of institutional momentum and epistemological style.
\end{historical}
