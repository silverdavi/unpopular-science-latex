\begin{historical}
Knot use predates written history. Archaeological finds of plant-fiber cordage with overhand knots span continents, indicating widespread functional use in early societies. As material technologies advanced, knots became essential in seafaring, shelter construction, and textile fabrication. By the 2\textsuperscript{nd} century AD, the physician Galen described surgical ligatures that relied on specific knot configurations, establishing a durable link between knots and applied biomechanics.

Systematic classification began in the late 19\textsuperscript{th} century. Peter Guthrie Tait, working on vortex dynamics, initiated a scheme for enumerating knot types based on minimal crossings. James Clerk Maxwell, in parallel, examined knotted field lines in his electromagnetic theory. These studies introduced the idea that knots could be analyzed as abstract topological entities, independent of material substrate.

By the early 20\textsuperscript{th} century, mathematical interest in knot equivalence intersected with empirical practice. Clifford W. Ashley’s 1944 reference work, \emph{The Ashley Book of Knots}, documented thousands of variants with practical commentary. Around the same time, J. J. Sylvester and others emphasized that friction, contact, and load path — factors absent in pure topology — govern real-world performance.

Late 20\textsuperscript{th}-century research brought knots into mechanical testing regimes. Investigations quantified how materials, crossing order, and dynamic tension affected strength and slippage. High-speed video and inertial measurement enabled detailed study of knot failure under cyclic loading.

In 2017, researchers at UC Berkeley published a dynamic model of shoelace failure. Using controlled experiments and slow-motion imaging, they demonstrated how walking - induced impacts and inertial lag of free ends initiate instability. Their findings linked knot topology to mechanical degradation over time, grounding abstract classifications in the failure modes of everyday structures.
\end{historical}
