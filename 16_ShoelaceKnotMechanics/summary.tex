The tendency for shoelaces to spontaneously untie stems from the interplay between knot topology and applied forces during walking. The common "granny knot" is topologically less stable than the "square knot," despite their superficial similarity. Which knot forms depends on the orientation of the initial crossing relative to subsequent looping motions. Walking generates cyclic impact forces and inertial oscillations that gradually loosen the weaker knot structure by allowing slack to propagate through the system. And perhaps — just perhaps — that's why children are convinced their parents tie better knots: a shoelace secured by Grandma, viewed from a frame with reversed chirality, might literally hold tighter.
