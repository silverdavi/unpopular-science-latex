\fullpageexercises{%
\textbf{Basic Exercises: Introduction to Knot Theory} \\[1em]
Knot theory is a branch of topology that studies closed loops embedded in three-dimensional space, considered up to continuous deformation without cutting or self-intersection. It is not about tying physical knots, but about understanding the abstract properties of such loops as mathematical objects.

\vspace{1em}

\textbf{1. Drawing the Unknot and Trefoil} \\[0.5em]
Draw two closed curves on paper: one that forms a simple loop with no crossings, and one that forms a trefoil knot — a closed loop with exactly three crossings. Indicate clearly which strand passes over and which passes under at each crossing. \\
\emph{Question:} Can the trefoil knot be deformed into a simple loop without cutting the curve?

\vspace{1em}

\textbf{2. Reidemeister Moves (Local Knot Manipulations)} \\[0.5em]
The Reidemeister moves are three local operations that preserve the topological type of a knot: \\
(I) Twist or untwist a loop. \\
(II) Add or remove two crossings that cancel each other. \\
(III) Slide a strand across a crossing. \\
\emph{Task:} On your drawing of the trefoil, apply each Reidemeister move once. Observe how the appearance changes without altering the knot's topology.

\vspace{1em}

\textbf{3. Crossing Number} \\[0.5em]
The crossing number of a knot is the smallest number of crossings needed in any planar projection of the knot. \\
\emph{Task:} Try to redraw the trefoil with fewer than three crossings. If not possible, explain why.

\vspace{1em}

\textbf{4. Mirror Images and Chirality} \\[0.5em]
Draw the mirror image of your trefoil knot by switching over-crossings to under-crossings and vice versa. \\
\emph{Question:} Can the trefoil and its mirror image be continuously deformed into each other without cutting?

\vspace{1em}

\textbf{5. Links vs. Knots} \\[0.5em]
A knot involves one loop; a link involves two or more loops. \\
\emph{Task:} Draw two loops that are interlocked (like chain links), and two loops that are separate. Without cutting, can one configuration be transformed into the other?

\vspace{1em}

\textbf{6. Counting Components} \\[0.5em]
Each knot or link has a number of connected components. \\
\emph{Task:} For each drawing, count the number of components. Identify which figures are knots and which are links.
}
