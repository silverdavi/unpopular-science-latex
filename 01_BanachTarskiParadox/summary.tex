The Banach–Tarski paradox shows that a solid sphere can be partitioned into finitely many disjoint pieces and, using only rigid motions, reassembled into two spheres identical to the original. This construction depends on the Axiom of Choice and the existence of non-measurable sets, whose behavior diverges from intuitions about volume. While not physically realizable, the result reveals how certain set-theoretic assumptions allow decompositions that defy standard notions of size and conservation.