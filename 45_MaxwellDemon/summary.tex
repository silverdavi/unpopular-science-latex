Maxwell's Demon, proposed in 1867, describes a thought experiment where a tiny being controls a door between two gas chambers, selectively allowing fast molecules into one chamber and slow ones into another. This sorting creates a temperature gradient from uniformity, seemingly decreasing entropy and violating the second law of thermodynamics. The resolution emerged through Landauer's principle (1961): the demon must erase information to continue sorting, and this erasure necessarily increases entropy by at least kB ln 2 per bit, offsetting the entropy reduction from sorting. This connection between information processing and thermodynamics reveals that information has physical embodiment with real thermodynamic consequences. The paradox remains relevant in quantum contexts, where measurement actively changes systems and quantum statistics alter the sorting possibilities.
