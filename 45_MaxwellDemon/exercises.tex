\fullpageexercises{%
\textbf{Energy Conservation Is an Amazing Tool: It Can Solve Seemingly Complex Problems Easily} \\[1em]
In classical mechanics, many systems that appear to require force analysis, Newton's laws, or torque computations can be solved using the principle of energy conservation. The following problems invite you to discover how straightforward solutions can emerge when total mechanical energy is conserved.

\vspace{1em}

\textbf{1. Rolling Sphere on an Incline} \\
A solid sphere of mass $m$ and radius $R$ is placed at the top of a frictional incline of angle $\theta$ and allowed to roll down without slipping. Using energy conservation (not Newton’s laws), determine the acceleration of the sphere’s center of mass. \\
\emph{Hint:} Total mechanical energy includes both translational and rotational kinetic energies.

\vspace{1em}

\textbf{2. Yo-Yo Drop} \\
A yo-yo of mass $m$ is held so that its string is taut and then released. The string unwinds without slipping as the yo-yo descends. The axle radius is $r$ and the moment of inertia about the center is $I$. Use energy conservation to determine the downward acceleration of the yo-yo’s center of mass. \\
\emph{Hint:} The yo-yo’s kinetic energy has both linear and rotational components; relate the angular speed to the linear speed via the axle radius.

\vspace{1em}

\textbf{3. Chain Falling Off a Table} \\
A uniform chain of linear mass density $\lambda$ lies coiled on a horizontal frictionless table. At time $t=0$, a small length starts to hang off the edge and the chain begins to slide off under gravity. Assuming no friction and no energy loss, use conservation of mechanical energy to find the acceleration of the chain as it falls. \\
\emph{Hint:} The center of mass of the hanging portion descends while its kinetic energy increases.

\vspace{1em}

\textbf{4. Man Walking on a Boat} \\
A man of mass $m$ walks a distance $d$ from one end of a boat of mass $M$ to the other. The boat floats on frictionless water. Determine how far and in what direction the boat moves relative to the water while the man walks. Use momentum conservation — no external horizontal forces act on the system. \\
\emph{Hint:} The center of mass of the system must remain fixed in the horizontal direction.


\noindent\hrulefill

\begin{center}
\rotatebox[origin=c]{180}{%
\begin{minipage}{0.95\textwidth}
\small
\noindent
\vspace{1em}
\(
\textbf{Answers: }\ 
\text{1: } a = \dfrac{5}{7}g\sin\theta \quad
\text{2: } a = \dfrac{mg}{m + \frac{I}{r^2}} \quad
\text{3: } a = \dfrac{g}{2} \quad
\text{4: } \text{Boat displacement } = -\dfrac{m}{M}d
\)
\end{minipage}
}
\end{center}


}
