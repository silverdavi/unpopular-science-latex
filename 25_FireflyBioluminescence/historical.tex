\begin{historical}
Fireflies have intrigued observers for millennia, with their rhythmic flashes illuminating summer landscapes and inspiring both folklore and scientific inquiry. Systematic investigation began in the early 20th century, culminating in Dr. E. Newton Harvey’s foundational studies in 1919, which linked the firefly's glow to a specific chemical reaction involving oxygen and a heatless form of combustion. His work established bioluminescence as a distinct physiological phenomenon and framed it within the broader study of oxidation reactions in living systems.

By the 1950s and 60s, researchers succeeded in isolating the key biochemical components: the substrate D-luciferin, the energy carrier ATP, and the enzyme luciferase, which catalyzes the light-producing reaction. These breakthroughs enabled direct experimentation on the reaction mechanism and launched decades of transdisciplinary work. Molecular biologists traced the genetic regulation of luciferase expression; biochemists elucidated its adenylation and oxidation kinetics; and physicists modeled the quantum transitions responsible for photon emission. The luciferase-luciferin system soon became a model for understanding energy conversion in biological systems — and later, a ubiquitous reporter in molecular biology.
\end{historical}


