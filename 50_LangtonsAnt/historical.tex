\begin{historical}
Christopher Gale Langton (born 1948/49) is widely recognized as one of the founding figures in the field of Artificial Life (ALife). Trained initially in physics and computer science, Langton became interested in the possibility of synthesizing life-like behaviors through computational systems rather than merely analyzing biological life. His work emphasized that certain organizational principles of life — adaptation, self-replication, robustness — might emerge from abstract dynamical processes, independent of the specific chemical substrates found in natural organisms.

In 1986, Langton introduced what would later become known as Langton's Ant while studying minimal computational systems capable of complex behavior. The ant was conceived as a concrete example of a broader research program: understanding how simple, local rules could give rise to large-scale order, unpredictability, and adaptive phenomena. Its dynamics were not engineered to solve a particular problem but to serve as a model system illustrating core concepts of emergence and self-organization.

Langton’s broader contributions to the field were formalized in 1987 with the organization of the first Workshop on the Synthesis and Simulation of Living Systems at Los Alamos National Laboratory, an event now recognized as the foundation of Artificial Life as a formal scientific discipline. His work on cellular automata, phase transitions, and computational complexity — particularly the introduction of the $\lambda$ parameter for classifying dynamical regimes — provided tools for systematically exploring the boundary between order and chaos in discrete systems.

The independent rediscovery of similar two-dimensional Turing machines by Greg Turk, referred to as "turmites," highlighted the naturalness of Langton's construction. The ant’s minimalism, universality, and emergent behavior positioned it as a central object of study not only within ALife but also in cellular automata theory, dynamical systems, and theoretical computer science.
\end{historical}
