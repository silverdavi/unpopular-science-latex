Langton's Ant exemplifies how minimal deterministic rules can generate complex, emergent behavior without external guidance. This system operates on a two-dimensional grid where a single agent follows two simple rules: on white squares, turn 90° clockwise, flip the color to black, and move forward; on black squares, turn 90° counterclockwise, flip to white, and advance. Despite this simplicity, the ant progresses through distinct phases — initially forming symmetric patterns, then exhibiting seemingly chaotic motion, before spontaneously constructing a repeating "highway" structure after prolonged evolution. The system bridges cellular automata (with parallel spatial updates) and Turing machines (with sequential computation), providing insight into computational universality and emergence.