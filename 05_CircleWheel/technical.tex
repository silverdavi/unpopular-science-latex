\begin{technical}
{\Large\textbf{The Phonological Evolution of Labiovelars in the Indo-European Descendants of \piefont{*kʷékʷlos}}}\\[0.7em]

\noindent\textbf{1. The PIE Labiovelars: Phonetic Properties and Distribution}\\[0.5em]
Proto-Indo-European (PIE) contained a series of labiovelar stops (\piefont{*kʷ}, \piefont{*gʷ}, \piefont{*gʷʰ}) characterized by simultaneous velar closure (tongue against the soft palate) and labialization (rounded lips). These contrasted with plain velars (\piefont{*k}, \piefont{*g}, \piefont{*gʰ}) and palatalized velars (\piefont{*ḱ}, \piefont{*ǵ}, \piefont{*ǵʰ}). 

The PIE root \piefont{*kʷékʷlos}, which reconstructs as a reduplicated form of \piefont{*kʷel-} ("to turn, rotate"), retained its labiovelars in early attested languages. Over time, different Indo-European branches underwent systematic shifts, reshaping the reflexes of \piefont{*kʷ} according to language-specific phonological rules.

\noindent\textbf{2. Greek: Labiovelar to Velar Shift}\\[0.5em]
By Mycenaean Greek (c. 1400 BCE, attested in Linear B), Greek had lost labiovelars in most environments, replacing them with plain velars. Thus, \piefont{*kʷ} became \piefont{*k}:
\[
\textgreek{κύκλος} (\emph{kyklos}) \quad \text{< PIE } \piefont{*kʷékʷlos}.
\]
This process, known as de-labialization, removed lip rounding. The same shift appears in:
\[
\textgreek{πέντε} (\emph{pente}, "five") < PIE \piefont{*pénkʷe}.
\]
Some Greek dialects, such as Boeotian, retained traces of labiovelars in specific phonological environments, but the standard evolution eliminated them.

\noindent\textbf{3. Sanskrit: Labiovelar to Palatal Shift}\\[0.5em]
In Vedic Sanskrit (attested c. 1200 BCE), PIE labiovelars merged with palatals before front vowels. This explains why \piefont{*kʷ} became \textsanskrit{च} (\emph{c}, pronounced [t͡ʃ]):
\[
\textsanskrit{चक्र} (\emph{chakra}) \quad \text{< PIE } \piefont{*kʷékʷlos}.
\]
This is part of a broader Indo-Iranian shift where labiovelars typically fronted or merged with palatal consonants, often conditioned by surrounding vowels.

\noindent\textbf{4. Latin: Conditional Reflexes in the Italic Branch}\\[0.5em]
In Latin (c. 700 BCE), the reflex of \piefont{*kʷ} depended on the following vowel:
- Before front vowels (\piefont{*e, *i}): \piefont{*kʷ} remained a labiovelar, later spelled \emph{qu}.
- Before back vowels (\piefont{*o, *u}): \piefont{*kʷ} lost its labialization and became \emph{c}.
- Before \piefont{*a}: Variation occurred, but de-labialization was common.

For example, in Latin \emph{colere} ("to cultivate"), the labiovelar \piefont{*kʷ} was retained before a back vowel:
\[
\text{PIE } \piefont{*kʷel-} \quad \to \quad \text{Latin \emph{colere}.}
\]
In contrast, in \emph{circulus} ("circle"), from PIE \piefont{*sker-}, the initial consonant was not a labiovelar but an inherited plain velar.

\noindent\textbf{5. Proto-Germanic: The Impact of Grimm’s Law}\\[0.5em]
By Proto-Germanic (c. 500 BCE), Grimm’s Law had altered the PIE stop system. Under this shift:
\[
\piefont{*kʷ} \quad \to \quad \piefont{*hw}.
\]
Thus, \piefont{*kʷékʷlos} became:
\[
\piefont{*hwehwlą} \quad \text{(Proto-Germanic)}
\]
which evolved into Old English \emph{hwēol} (c. 700 CE), Middle English \emph{whele}, and Modern English \emph{wheel}. Other Germanic languages followed similar transformations, such as Old Norse \emph{hjól}.

\noindent\textbf{6. Summary of Phonological Shifts}\\[0.5em]
\begin{itemize}
    \item Greek: \piefont{*kʷ} > \piefont{k} (de-labialization) → \textgreek{κύκλος} (\emph{kyklos}).
    \item Sanskrit: \piefont{*kʷ} > \piefont{c} (palatalization) → \textsanskrit{चक्र} (\emph{chakra}).
    \item Latin: \piefont{*kʷ} retained as \emph{qu} or de-labialized → \emph{colere, circulus}.
    \item Germanic: \piefont{*kʷ} > \piefont{hw} (Grimm’s Law) → Proto-Germanic \piefont{*hwehwlą} → Old English \emph{hwēol} → Modern English \emph{wheel}.
\end{itemize}

\noindent
These transformations illustrate how a single PIE labiovelar stop produced diverse reflexes across Indo-European languages, shaping words that remain etymologically linked despite significant phonetic divergence.

\vspace{0.5em}
\noindent\textbf{References}\\
Fortson, B. (2010). \emph{Indo-European Language and Culture: An Introduction}. Wiley-Blackwell.\\
Ringe, D. (2006). \emph{From Proto-Indo-European to Proto-Germanic}. Oxford University Press.\\
Beekes, R. (2011). \emph{Etymological Dictionary of Greek}. Brill.\\
Online Etymology Dictionary: \url{https://www.etymonline.com/}\\
\end{technical}
