\begin{tcolorbox}[
    colback=white,
    colframe=gray!40,
    boxrule=0.5pt,
    arc=1mm,
    left=10pt,
    right=10pt,
    top=10pt,
    bottom=10pt,
    width=\textwidth,
    enlarge left by=0mm,
    sharp corners=south,
    breakable
]
\setlength{\parskip}{1em}

This book is not a popular science book. It is not a textbook. It is not an academic book. It is not even a chimera of the above.  

It does share some goals with the three: to inspire wonder (like a popular science book), to include some rigour (like textbooks), and to introduce readers to phenomena that might challenge their understanding (as academic works often achieve).

As with some combinations, like a sushi-pizza restaurant, it excels at none.  
You'll find that the main exposition is not long enough to really understand the subject, the technical part is either too abstract or too detailed to actually follow, and even though I claim rigour, mostly von Neumann's line that \textit{there's no sense in being precise when you don't know what you're talking about} fits too well.

But, if my plan works, you'll get the appetite to routinely leave this sushi-pizza diner and cross the street. Maybe to a fine-dining textbook because you're intrigued and want to solve it properly. Maybe to a street seller of roasted chestnuts, to browse Wikipedia or blogs and get more context for this absolutely fantastical world I'm trying to share with you.

All of the bits I'm sharing, except for a few that friends recommended, have a personal story. I remember how I learned about them, and I'm genuinely excited to write them down.  

I wish I could infect you with some of that excitement.

Here we go.
\end{tcolorbox}


\begin{tcolorbox}[
    colback=red!5,
    colframe=red!60!black,
    boxrule=1pt,
    arc=0.5mm,
    left=10pt,
    right=10pt,
    top=10pt,
    bottom=10pt,
    width=\textwidth,
    sharp corners=south,
    breakable,
    title=\textbf{DRAFT WARNING}
]
\setlength{\parskip}{1em}

This is a \textbf{very early draft}.\\
Parts of it are placeholders.\\
Some claims may be wrong.\\
Sections may vanish without notice.\\
Reader discretion is advised.

\begin{center}
{\addfontfeatures{LetterSpace=5.0}\scshape Manus adhuc in pulvere}\\
\textit{The hands are still in the dust.}
\end{center}

\end{tcolorbox}
