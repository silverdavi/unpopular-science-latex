Zero-knowledge proofs enable verifying the truth of a statement without revealing any information beyond its validity. These cryptographic protocols satisfy three properties: completeness (valid statements can be proven), soundness (false statements cannot be proven), and zero-knowledge (the proof reveals nothing about the statement except its truth). The mechanisms rely on cryptographic commitments, challenge-response interactions, or probabilistic verification that make it computationally infeasible to succeed without knowing the secret. Applications include authentication without password transmission, privacy-preserving transactions, and confidential verification of regulatory compliance.