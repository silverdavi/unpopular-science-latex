
\begin{historical}
The formal development of zero-knowledge proofs (ZKPs) emerged in the 1980s at the intersection of cryptography and complexity theory. Shafi Goldwasser, Silvio Micali, and Charles Rackoff introduced the concept in their groundbreaking 1985 paper, initially in the context of interactive proofs. At the time, the focus of cryptographic research was shifting from merely encrypting information to controlling what could be deduced from a communication. Traditional encryption protected secrets, but ZKPs introduced a new paradigm: verifiable secrecy.
The intuition behind ZKPs predates digital systems. In folklore, riddles and games often involved demonstrating hidden knowledge without disclosure. But formalizing this intuition required the language of computational hardness and probabilistic verification. The invention of ZKPs set off a cascade of innovations in cryptographic protocols — eventually enabling anonymous authentication, blockchain privacy, and secure computation. From paper to protocol, the ability to convince without revealing would redefine digital trust.
\end{historical}