\begin{historical}
In 1933, Swiss astrophysicist Fritz Zwicky analyzed the velocity dispersion of galaxies in the Coma Cluster and found their motions to be too fast to be gravitationally bound by the visible mass alone. He introduced the term “dunkle Materie” (dark matter) to describe the missing component. Though initially met with skepticism, his mass discrepancy hinted at a deeper problem in astrophysical mass accounting.

The issue resurfaced decisively in the 1970s when Vera Rubin and Kent Ford measured rotation curves of spiral galaxies using optical spectroscopy. Instead of decreasing with radius as expected from luminous matter distributions, the rotation speeds remained flat well beyond the visible edge. Independent radio observations by Albert Bosma confirmed this effect through 21-cm emission from neutral hydrogen, revealing a pervasive halo of unseen mass enveloping each galaxy.

By the early 1980s, theorists such as Jeremiah Ostriker and Jim Peebles emphasized the necessity of dark matter to explain large-scale structure formation. Without a non-luminous component, galaxies and clusters could not form on observed timescales. Theoretical simulations incorporating dark matter successfully reproduced the filamentary distribution of galaxies seen in redshift surveys.

Gravitational lensing provided additional, independent confirmation. Light from distant sources bent around foreground mass concentrations showed more deflection than visible matter alone could account for. In 2006, observations of the Bullet Cluster — a high-speed collision of galaxy clusters — visually separated dark matter from hot gas via X-ray and lensing data, offering direct evidence of a collisionless mass component.

Dark matter’s influence now spans cosmology, astrophysics, and particle physics. Measurements of cosmic microwave background anisotropies by missions such as WMAP and Planck established dark matter as essential for matching early-universe fluctuations to present-day structure. Though its particle identity remains unknown, dark matter continues to unify diverse strands of observational evidence under a coherent gravitational framework.
\end{historical}
