General relativity and quantum field theory represent fundamentally incompatible frameworks for describing physical reality. GR portrays gravity as spacetime curvature — a geometric property emerging from mass-energy distribution — while QFT describes forces through particle exchange on a fixed background. This conceptual divide manifests in specific problems: the cosmological constant discrepancy (120 orders of magnitude between QFT vacuum energy predictions and observed values), the non-renormalizability of gravity (preventing standard quantum treatment), and the black hole information paradox (challenging quantum unitarity). These conflicts reveal the limits of our current understanding and motivate ongoing research into quantum gravity theories that might resolve these theoretical inconsistencies.
