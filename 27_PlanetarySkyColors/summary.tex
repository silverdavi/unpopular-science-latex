The colors observed in astronomical objects originate from defined physical interactions between light and matter. Earth's blue sky results from Rayleigh scattering, where atmospheric molecules preferentially scatter shorter wavelengths. This principle extends across astronomy, where spectral modifications encode properties like temperature, composition, and motion. Stars emit thermal radiation with characteristic spectra determined by temperature — hotter stars appear blue, cooler stars red. Planets derive colors from reflection, absorption, and scattering processes in their atmospheres and surfaces. Nebulae display colors through emission from ionized gas, reflection of starlight by dust, or absorption creating dark silhouettes.