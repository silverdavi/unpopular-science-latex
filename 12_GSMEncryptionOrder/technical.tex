\begin{technical}
{\Large\textbf{Toy GSM‑Style Frame: How Post‑Encoding XOR Leaks Keystream}}\\[0.7em]

\noindent\textbf{1.\,Frame Layout}\\[0.5em]
A single voice frame carries \textbf{8 information bits} $s_1,\dots,s_8$.  
GSM inserts fixed training bits and applies forward‑error correction; we model that with a minimal, fully visible layout:
\[
\underbrace{\color{gray}{1\,1\,0\,0}}_{\text{training}}\;
\underbrace{s_1\,s_2}_{\text{speech}}\;
\underbrace{\color{gray}{1\,0}}_{\text{pad}}\;
\underbrace{s_3\,s_4}_{\text{speech}}\;
\underbrace{\color{gray}{0\,1}}_{\text{pad}}\;
\underbrace{s_5\,s_6}_{\text{speech}}\;
\underbrace{\color{gray}{1\,1}}_{\text{parity}}\;
\underbrace{s_7\,s_8}_{\text{speech}}
\tag{1}
\]
Exactly \(\mathbf{24}\) bits form the encoder input: 8 unknown speech bits and 16 deterministic bits known to the attacker.

\medskip
\noindent\textbf{2.\,Redundancy (Mini‑Convolutional Code)}\\[0.5em]
Each input bit \(x_i\) passes through a toy \((1,1/2)\) convolutional code with generator polynomials \((1,\;1+D)\):
\[
y_{i,0}=x_i,
\qquad
y_{i,1}=x_i \oplus x_{i-1}.
\]
This yields 48 output bits \(Y=[y_0,\dots,y_{47}]\).  Every pair satisfies
\[
y_{i,1}\;\oplus\;y_{i-1,0}=y_{i,0}
\quad\forall i,
\tag{2}
\]
a parity relation that later survives encryption.

\medskip
\noindent\textbf{3.\,Interleaver}\\[0.5em]
A fixed block interleaver permutes the 48 bits:
\[
\pi(i)=\bigl(i\bmod4\bigr)\cdot12+\left\lfloor\frac{i}{4}\right\rfloor,
\tag{3}
\]
a public mapping known to attacker and receiver.

\medskip
\noindent\textbf{4.\,Encryption After Coding}\\[0.5em]
Encryption XORs a keystream \(K_0,\dots,K_{47}\) with the permuted code bits:
\[
C_i = Y_{\pi(i)} \oplus K_i.
\tag{4}
\]
Because XOR preserves length and position, all structure in \(Y\) remains in masked form in \(C\).

\bigskip
\noindent\textbf{5.\,Ciphertext‑Only Attack Sketch}\\[-0.1em]
\begin{enumerate}[label=\arabic*., leftmargin=1.6em]
  \item \textit{Training leakage}. From known training/pad/parity bits the attacker obtains
        \(K_i = C_i \oplus Y_{\pi(i)}\) for 14 positions.
  \item \textit{Parity recursion}.  Using Equation (2) these known~\(K_i\) values let the
        attacker deduce additional \(Y_j\) and thus more \(K_j\), cascading until
        the full keystream is known.
  \item \textit{Speech recovery}.  With all \(K_i\) recovered, the attacker inverts
        the interleaver and convolutional code to extract \(s_1,\dots,s_8\).
\end{enumerate}
In simulation, \textbf{14 known ciphertext bits} suffice to crack the entire 48‑bit keystream.

\bigskip
\noindent\textbf{6.\,Why Encrypt‑First Stops the Leak}\\[0.5em]
If encryption preceded coding, the encoder would process \(X \oplus K'\) rather than \(X\).
Parity relation (2) would then bind unknown values, blocking the attack. Fixed
fields would reveal nothing until after decryption.

\medskip
\noindent\textbf{Take‑away}\,: Encryption applied \emph{after} structure preserves that structure
in the ciphertext.  In GSM the predictable layout of training, padding, and coding
redundancy leaks keystream fragments, which in turn expose the full A5/1 state.

\vspace{0.6em}
\noindent\textbf{Reference}\\[0.3em]
{\small
Barkan, E.\;Biham, E.\;Keller, N.\ “Instant Ciphertext‑Only Cryptanalysis of GSM Encrypted Communication.” \textit{Journal of Cryptology} 21 (3), 392–429 (2008).}
\end{technical}
