\begin{historical}
In the late 1990s, two independent collaborations studying distant Type Ia supernovae — the Supernova Cosmology Project led by Saul Perlmutter and the High-Z Supernova Search Team led by Adam Riess and Brian Schmidt — reported that these stellar explosions were fainter than anticipated. Their apparent brightness suggested they were farther away than a purely decelerating cosmological model would indicate. This unexpected outcome signaled an accelerating rate of cosmic expansion. 

Early theoretical groundwork on cosmic expansion stemmed from solutions to Einstein’s field equations by Alexander Friedmann in the 1920s and Georges Lemaître in the 1930s. Although Einstein initially introduced a cosmological constant $\Lambda$ to force a static universe, Hubble’s observations of galactic recession (1929) established expansion as a fact. Decades later, detailed measurements of supernova luminosities confirmed that this expansion was not slowing down, but accelerating.

Recognition of this discovery came in 2011, when the Nobel Prize in Physics was awarded to Perlmutter, Riess, and Schmidt for uncovering cosmic acceleration. Additional evidence arose from measurements of cosmic microwave background anisotropies and large-scale galaxy distributions, all converging on the conclusion that the expansion of the universe is not merely continuing but speeding up.
\end{historical}
