\begin{technical}
{\Large\textbf{Cosmic Acceleration and Dark Energy}}

Evidence for accelerating cosmic expansion emerged in the late 1990s from observations of Type Ia supernovae. These supernovae serve as \emph{standard candles} because their peak luminosities are nearly uniform, a result of thermonuclear ignition when a white dwarf reaches a critical mass. Comparing their observed brightness to the known intrinsic value yields a distance estimate. Redshift measurements then indicate how much the Universe has expanded during the light’s journey, mapping the evolution of the scale factor \(a(t)\). The scale factor quantifies the relative expansion of space as a function of time: if two galaxies are separated by a comoving distance \(r\), their physical distance at time \(t\) is given by \(d(t) = a(t) \cdot r\).

Unexpectedly, the data showed that distant Type Ia supernovae appeared dimmer than predicted under a decelerating Universe. This implies they are farther away than expected, indicating that cosmic expansion has been accelerating over the past several billion years.

\medskip

\noindent\textbf{Friedmann Equation and the Role of Pressure}\\
Within general relativity, the dynamics of cosmic expansion are governed by the Friedmann equation:
\begin{equation}
  H^2(t) \;=\; \left(\frac{\dot{a}}{a}\right)^2 \;=\; \frac{8\pi G}{3}\,\rho_{\text{tot}} \;+\; \frac{\Lambda c^2}{3}.
\end{equation}
Here:
\begin{itemize}[leftmargin=*]
  \item \(H(t) = \dot{a}/a\) is the Hubble parameter.
  \item \(\rho_{\text{tot}}\) is the total energy density of the Universe (including matter, radiation, etc.).
  \item \(\Lambda\) is the cosmological constant.
\end{itemize}

To understand acceleration, one must consider the second Friedmann equation, which relates the acceleration of the scale factor to pressure:
\begin{equation}
\frac{\ddot{a}}{a} = -\frac{4\pi G}{3} \left( \rho + \frac{3P}{c^2} \right).
\end{equation}

This expression shows that pressure contributes to the expansion dynamics alongside energy density. For matter and radiation, which satisfy \(P \geq 0\), the result is deceleration: gravity dominates and slows the expansion. However, if a component exists with sufficiently negative pressure — specifically, \(P < -\rho c^2/3\) — then the right-hand side becomes positive and \(\ddot{a} > 0\). In this regime, the expansion accelerates.


A cosmological constant \(\Lambda\) corresponds to a constant energy density with pressure \(P = -\rho c^2\), which satisfies this condition. In physical terms, negative pressure acts as a repulsive gravitational source, driving space itself to expand at an accelerating rate.

\medskip

\noindent\textbf{Dark Energy and the \(\Lambda\)CDM Model}\\
Incorporating \(\Lambda\) as a form of dark energy, the resulting model—known as \(\Lambda\)CDM—successfully fits a wide range of observational data. Measurements of the cosmic microwave background (CMB), baryon acoustic oscillations, and galaxy clustering support a Universe composed of:
\[
\begin{aligned}
\text{Dark energy:} \quad &\sim 68\% \\
\text{Dark matter:} \quad &\sim 27\% \\
\text{Baryonic matter:} \quad &\sim 5\%.
\end{aligned}
\]

High-redshift supernovae, such as SN 1997ff at \(z \approx 1.7\), confirm a transition from a decelerating phase (matter-dominated) to the current accelerating phase (dark-energy-dominated), consistent with predictions from \(\Lambda\)CDM.

\medskip

\noindent\textbf{Outstanding Issues: Vacuum Energy and \(H_0\) Tension}\\
Despite its empirical success, the physical nature of dark energy remains unknown. The value of \(\Lambda\) inferred from observations is smaller than naive quantum field theory predictions for vacuum energy by 120 orders of magnitude—a discrepancy sometimes described as the worst theoretical prediction in physics.

Further complications arise from the so-called Hubble tension: early-universe data from the Planck satellite (based on the CMB) yield a value of the Hubble constant \(H_0 \approx 67.4\,\text{km/s/Mpc}\), whereas local distance ladder methods yield \(H_0 \approx 73\,\text{km/s/Mpc}\). This discrepancy, now exceeding 4σ, may signal unaccounted-for systematics or point toward new physics.

Proposed resolutions include early dark energy (a transient acceleration phase prior to recombination), modifications to recombination physics, or revised calibrations of Cepheid-based distances. None has yet resolved the tension conclusively.

\medskip

\noindent\textbf{References} \\
Riess, A. G. et al. (1998). Observational Evidence from Supernovae for an Accelerating Universe. \textit{AJ}, 116, 1009.\\
Perlmutter, S. et al. (1999). Measurements of \(\Omega\) and \(\Lambda\) from 42 High-Redshift Supernovae. \textit{ApJ}, 517, 565.\\
Planck Collaboration (2020). Planck 2018 results. \textit{A\&A}, 641, A6.\\
Di Valentino, E. et al. (2021). In the realm of the Hubble tension—a review of solutions. \textit{Class. Quant. Grav.}, 38, 153001.
\end{technical}
