\begin{historical}
In the mid-19th century, Hermann Grassmann established a foundation for vector algebra that influenced how mathematicians approached geometric concepts in higher dimensions. Karl Weierstrass later advanced the formal treatment of bilinear forms, paving the way for broader studies of inner products.

By the early 20th century, David Hilbert and John von Neumann had built on these ideas to construct the rigorous framework of Hilbert spaces, which became a cornerstone of quantum mechanics. In the 1930s, Paul Dirac’s bra-ket notation offered a succinct way to represent quantum states, making the geometric relationships between subspaces more transparent.

During the 1950s, Fritz John and Tosio Kato investigated principal angles between subspaces, refining how mathematicians quantified overlaps and correlations within Hilbert spaces. These tools, closely tied to the geometry of quantum systems, gained increasing relevance when Richard Feynman proposed the concept of a quantum simulator in 1982, hinting at computational methods that exploit these angular relationships.

Throughout the 1990s, interest in quantum information surged with milestones such as Peter Shor’s 1994 factoring algorithm, which relied on precise control over quantum states and their subspaces. As the field moved into the 21st century, angle-based methods became integral to entanglement studies, quantum error correction, and the design of efficient quantum algorithms. Today, they remain indispensable for researchers aiming to harness the computational power of quantum systems and deepen our understanding of the geometry underlying quantum theory.
\end{historical}