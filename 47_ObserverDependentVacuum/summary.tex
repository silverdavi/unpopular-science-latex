The vacuum's nature transforms across physical theories, from the classical notion of emptiness to the observer-dependent construct of quantum field theory in curved spacetime. In flat spacetime, quantum fields fluctuate around minimum energy states, yet all inertial observers agree on particle absence. This consensus breaks down in curved spacetime or for accelerating observers, where the definition of particles depends on the observer's trajectory. The Unruh effect exemplifies this: an observer accelerating through empty space perceives a thermal bath of particles at temperature proportional to their acceleration (T = ħa/2πckB). This occurs because acceleration alters how field modes separate into positive and negative frequencies, transformed by Bogoliubov coefficients that mix creation and annihilation operators.