\begin{historical}
Isaac Newton’s 1687 \textit{Principia} described gravity as a force acting at a distance between masses, accurately predicting the behavior of falling objects and planetary orbits. This framework dominated physics for more than two centuries. However, increasingly precise astronomical measurements revealed small but persistent anomalies, such as the unexplained precession of Mercury’s orbit.

In 1915, Albert Einstein introduced general relativity, reframing gravity as the curvature of spacetime rather than a force. This new theory predicted phenomena beyond Newtonian physics, including gravitational time dilation and the deflection of light by massive bodies.

Experimental confirmation followed. In 1919, Arthur Eddington observed starlight bending around the Sun during a solar eclipse — the first direct observational support for general relativity. In 1959, the Pound–Rebka experiment measured gravitational redshift using gamma rays, demonstrating that time runs differently in stronger gravitational fields. Further tests strengthened these results: in 1976, Gravity Probe A used a hydrogen maser clock launched on a suborbital rocket to verify altitude-dependent time dilation, and in 1980, cesium atomic clocks at Germany’s Physikalisch-Technische Bundesanstalt measured the same effect with even higher precision.

By the late 20th century, relativistic corrections had become indispensable to modern technology. The Global Positioning System (GPS), for example, relies on continuous adjustments for both special and general relativistic time effects to maintain navigational accuracy.

In 2015, the Laser Interferometer Gravitational-Wave Observatory (LIGO) made the first direct detection of gravitational waves — ripples in spacetime caused by the merger of two black holes over a billion light-years away. This observation confirmed a key prediction of general relativity in the strong-field regime. Subsequent detections by LIGO and the Virgo interferometer have continued to validate the theory with remarkable precision, including observations of neutron star collisions and black hole mergers.

Today, general relativity stands as one of the most thoroughly tested and confirmed theories in physics. Its predictions have been validated across a vast range of scales, from subatomic experiments to cosmological observations, reinforcing its role in our understanding of gravity.
\end{historical}
