General relativity predicts drastically different perspectives for observers watching an object fall into a black hole versus the falling object itself. External observers see the object progressively slow, redden, and dim as it approaches the event horizon, appearing to freeze at the boundary due to extreme gravitational time dilation. In contrast, the falling object experiences no unusual effects when crossing the event horizon, continuing inward without local indication of having passed this critical boundary. This disparity in observations illustrates how spacetime curvature alters the relationship between reference frames, arresting intuitions about simultaneity and time.
