\begin{technical}
{\Large\textbf{Spectre and Meltdown: Mechanics and Leakage Rates}}\\[0.7em]

\textbf{Introduction}\\[0.5em]
Spectre and Meltdown exploit transient execution: instructions issued before permissions or branches resolve. Although architecturally squashed, these instructions leak data through persistent microarchitectural side effects — most notably, the cache — observable via timing measurements.

\vspace{0.4em}
\textbf{Meltdown: Transient Load + Fault Deferral}\\[0.5em]
The CPU speculatively executes a faulting memory load and uses the result before the exception is raised. A dependent load encodes the secret byte into cache state.

\begin{verbatim}
; RCX ← kernel memory address
mov al, [rcx]
shl rax, 12
mov rbx, [probe + rax]
\end{verbatim}
\vspace{-0.5em}
\textit{Comments:}
\begin{itemize}[noitemsep,topsep=0pt]
\item Line 1: Transiently loads a protected byte into AL.
\item Line 2: Multiplies the secret by 4096 (page alignment).
\item Line 3: Loads from \texttt{probe[s × 4096]}, caching a secret-dependent line.
\end{itemize}

\textit{Observation:} The attacker probes access times to \texttt{probe[i × 4096]} and identifies the secret by locating the cache hit.

\vspace{0.4em}
\textbf{Spectre: Mistrained Bounds Bypass}\\[0.5em]
The attacker mistrains the branch predictor to speculatively skip a bounds check. This leads to out-of-bounds access during the transient window.

\begin{verbatim}
if (x < array_length)
    temp = probe[secret[x] * 4096];
\end{verbatim}
\vspace{-0.5em}
\textit{Only executed speculatively.} The loaded cache line leaks \texttt{secret[x]}.

\textit{Setup:} Train with in-bounds \( x \); switch to out-of-bounds. Side effects persist even when the branch is mispredicted.

\vspace{0.4em}
\textbf{Cache Timing Oracle}\\[0.5em]
Let \( S \in \{0,\dots,255\} \) be the secret. Let \( C_i \) be the access time to \texttt{probe[i × 4096]}. Then:
\[
\begin{aligned}
\hat{S} &= \arg\min_i C_i \\
\text{where} \quad
C_i &= 
\begin{cases}
T_\text{hit} + \epsilon_i & \text{if } i = S \\
T_\text{miss} + \epsilon_i & \text{otherwise}
\end{cases}
\end{aligned}
\]
With low noise, each measurement leaks approximately 8 bits.

\vspace{0.4em}
\textbf{Information-Theoretic View}\\[0.5em]
Let \( X \) be the secret, \( Y \) the timing observation. Leakage is given by:
\[
I(X; Y) = H(X) - H(X \mid Y)
\]
Under uniform \( X \) and low timing noise, \( I(X; Y) \to 8 \) bits. Repetition and majority decoding mitigate jitter and measurement error.

\vspace{0.4em}
\textbf{Measured Bandwidth}\\[0.5em]
\begin{tabular}{@{}ll@{}}
Meltdown (local):     & 3.2–503 KB/s \\
Spectre (native):     & ~10 KB/s \\
Spectre (JavaScript): & ~1–2 bits/sec \\
\end{tabular}

\vspace{0.5em}
\textbf{References}\\
Kocher et al. (2018). \textit{Spectre Attacks: Exploiting Speculative Execution}. arXiv:1801.01203.\\
Lipp et al. (2018). \textit{Meltdown: Reading Kernel Memory from User Space}. USENIX Security.\\
Bernstein, D. (2005). \textit{Cache-Timing Attacks on AES}.\\
Shannon, C. (1948). \textit{A Mathematical Theory of Communication}. Bell System Technical Journal.
\end{technical}
