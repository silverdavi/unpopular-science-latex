\begin{technical}
{\Large\textbf{Permutation Inversion under Irreversible Transpositions}}\\[0.7em]

\noindent\textbf{Setup and Constraint}\\[0.5em]
Let \( A = \{1, 2, \dots, n\} \) be a finite set. Let \( P \in S_n \) be a permutation composed of distinct transpositions:
\[
P = \tau_1 \circ \tau_2 \circ \dots \circ \tau_m, \quad \tau_i = (a_i\, b_i), \quad (a_i\, b_i) \neq (a_j\, b_j) \text{ for } i \neq j.
\]
Let \( x, y \notin A \) be two auxiliary elements. Define the augmented set \( A^+ = A \cup \{x, y\} \). The goal is to construct \( \sigma \in S_{n+2} \), composed entirely of transpositions involving either \( x \) or \( y \), such that:
\[
\sigma \circ P = \text{id}_{A}, \quad \text{and} \quad \sigma \text{ uses no transposition from } \{\tau_1, \dots, \tau_m\}.
\]

\noindent\textbf{Cycle Decomposition}\\[0.5em]
Every permutation \( P \in S_n \) admits a unique decomposition into disjoint cycles:
\[
P = C_1 \circ C_2 \circ \dots \circ C_r,
\]
where each \( C_i = (a^{(i)}_1\, a^{(i)}_2\, \dots\, a^{(i)}_{k_i}) \) is a \( k_i \)-cycle with \( a^{(i)}_j \in A \), and all elements of \( A \) appear in exactly one cycle.

\noindent\textbf{Cycle Reversal Construction}\\[0.5em]
To reverse a single cycle \( C = (a_1\, a_2\, \dots\, a_k) \), define a sequence of transpositions involving \( x \) and \( y \) only:
\begin{align}
\alpha &= (x\, a_1)(x\, a_2)\dots(x\, a_k), \\
\beta  &= (y\, a_k)(y\, a_{k-1})\dots(y\, a_2), \\
\gamma &= (x\, a_1), \\
\delta &= (y\, a_1),
\end{align}
and define
\begin{align}
\sigma_C &= \delta \circ \gamma \circ \beta \circ \alpha.
\end{align}

Applying \( \sigma_C \) to \( C \) yields the identity permutation on \( \{a_1, \dots, a_k\} \), i.e.,
\[
\sigma_C \circ C = \text{id}_{\{a_1, \dots, a_k\}}.
\]
The effect on \( x \) and \( y \) is a transposition:
\[
\sigma_C \circ C = (x\, y) \quad \text{on } \{x, y\}.
\]

\noindent\textbf{Global Reversal Strategy}\\[0.5em]
Let \( P = C_1 \circ C_2 \circ \dots \circ C_r \) as above. For each cycle \( C_i \), construct \( \sigma_{C_i} \) using the method above. Define
\begin{align}
\sigma &= \sigma_{C_1} \circ \sigma_{C_2} \circ \dots \circ \sigma_{C_r}.
\end{align}
Then
\[
\sigma \circ P =
\begin{cases}
\text{id}_{A \cup \{x, y\}} & \text{if } r \text{ is even}, \\
(x\, y) & \text{if } r \text{ is odd}.
\end{cases}
\]
If \( r \) is odd, apply a final correction:
\[
\sigma' = (x\, y) \circ \sigma.
\]

\noindent\textbf{Swap Count}\\[0.5em]
Each \( \sigma_{C_i} \) involves \( 2k_i + 2 \) transpositions, where \( k_i = |C_i| \). The total number of swaps (excluding the final correction if needed) is:
\begin{align}
\sum_{i=1}^{r} (2k_i + 2) &= 2 \sum_{i=1}^{r} k_i + 2r = 2n + 2r.
\end{align}
If \( r \) is odd, the final correction adds 1, yielding:
\[
\text{Total swaps} = 2n + 2r + 1.
\]

\noindent\textbf{Example: 3-Cycle Inversion}\\[0.5em]
Let \( C = (1\,2\,3) \). Apply the construction:
\begin{align*}
\alpha &= (x\,1)(x\,2)(x\,3), \\
\beta  &= (y\,3)(y\,2), \\
\gamma &= (x\,1), \\
\delta &= (y\,1),
\end{align*}
and define
\[
\sigma_C = (y\,1)(x\,1)(y\,3)(y\,2)(x\,1)(x\,2)(x\,3).
\]
Then
\begin{align}
\sigma_C \circ (1\,2\,3) &= \text{id}_{\{1,2,3\}}, \\\quad \text{and } \sigma_C \circ (1\,2\,3) &= (x\, y) \text{ on } \{x, y\}.
\end{align}

\vspace{0.5em}
\noindent\textbf{References:}\\
Keeler, K. (2010). Mind-switching and Permutation Inversion with Constraints.\\
Evans, R., Huang, H., Nguyen, T. (2016). \textit{Keeler’s Theorem and Products of Distinct Transpositions}. UCSD Mathematics Department.
\end{technical}
